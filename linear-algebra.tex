\documentclass[12pt, a4paper]{report}
\usepackage[left=3.0cm, right=1.5cm, vmargin=2.0cm]{geometry} %ГОСТ 7.32
\usepackage{indentfirst}
\usepackage{setspace}
\onehalfspacing
%--------------------------------------------------------------------------------------------------------------------------
\usepackage[T2A]{fontenc}
\usepackage[utf8]{inputenc}
\usepackage[russian]{babel}
\usepackage{csquotes}
\usepackage{hyphenat}
\usepackage[a]{esvect}
\hyphenation{ма-те-ма-ти-ка вос-ста-нав-ли-вать}
%--------------------------------------------------------------------------------------------------------------------------
\usepackage[hidelinks]{hyperref}
\usepackage[intoc, russian]{nomencl}
\makenomenclature
\usepackage{graphicx}
\graphicspath{ {d:/texstudio-pics/} }
\usepackage[dvipsnames]{xcolor}
\usepackage{amssymb}
\usepackage{mathtools, bm, accents}
\usepackage[makeroom]{cancel}
\usepackage{upgreek}
\usepackage{enumerate}
\usepackage{xfrac}
%--------------------------------------------------------------------------------------------------------------------------
\makeatletter
\newsavebox\myboxA
\newsavebox\myboxB
\newlength\mylenA

\newcommand*\xoverline[2][0.75]{%
	\sbox{\myboxA}{$\m@th#2$}%
	\setbox\myboxB\null% Phantom box
	\ht\myboxB=\ht\myboxA%
	\dp\myboxB=\dp\myboxA%
	\wd\myboxB=#1\wd\myboxA% Scale phantom
	\sbox\myboxB{$\m@th\overline{\copy\myboxB}$}%  Overlined phantom
	\setlength\mylenA{\the\wd\myboxA}%   calc width diff
	\addtolength\mylenA{-\the\wd\myboxB}%
	\ifdim\wd\myboxB<\wd\myboxA%
	\rlap{\hskip 0.5\mylenA\usebox\myboxB}{\usebox\myboxA}%
	\else
	\hskip -0.5\mylenA\rlap{\usebox\myboxA}{\hskip 0.5\mylenA\usebox\myboxB}%
	\fi}
\makeatother
%--------------------------------------------------------------------------------------------------------------------------
\usepackage{xstring}
\newcommand{\vect}[3][n]{\IfEqCase{#1}%
										{%
											{l}{\vv{\mathbf{#2}}}%
											{l*}{\vv*{\mathbf{#2}}{#3}}%
											{c}{\underaccent{\boldsymbol{\downarrow}}{\mathbf{#2}}_{#3}}%
											{n}{\mathbf{#2}_{#3}}%
										}}%
%--------------------------------------------------------------------------------------------------------------------------
\makeatletter
\newenvironment{union}{%
	\matrix@check\union\env@union
}{%
	\endarray\right.%
}
\def\env@union{%
	\let\@ifnextchar\new@ifnextchar
	\left\lbrack
	\def\arraystretch{1.2}%
	\array{@{}l@{\quad}l@{}}%
}
\makeatother
%--------------------------------------------------------------------------------------------------------------------------
\usepackage{tikz}
	\usetikzlibrary{cd, matrix, fit}
\newcommand\encircle[1]{\tikz[baseline=(X.base)]\node(X)[draw, shape=circle, inner sep=2pt] {#1};}
\newcommand\ensq[1]{\tikz[baseline=(X.base)]\node(X)[draw, shape=rectangle, inner sep=3pt] {#1};}
\newcommand{\tikzmark}[2]{\tikz[remember picture,baseline=(#1.base)]\node[circle, dashed,black,draw,text=black,anchor=center,inner sep=1pt] (#1) {$#2$};}
\newcommand{\tikzmarkempty}[2]{\tikz[overlay,remember picture,baseline=(#1.base)] \node (#1) {$#2$};}
\newcommand{\tikzmarkemptyX}[2]{\tikz[overlay,remember picture,baseline=(#1.base)] \node [inner sep=1pt] (#1) {$#2$};}
%--------------------------------------------------------------------------------------------------------------------------
\newcommand{\df}[1][]{\begin{flushleft}\textbf{\underline{Опр} \textcolor{Plum}{#1}}\end{flushleft}}
\newcommand{\ex}{\begin{flushleft}\textbf{\underline{Пр}}\end{flushleft}}
\newcommand{\lm}[2][]{\begin{flushleft}\textbf{\ensq{Л\(^\mathbf{#1}\)} \textcolor{Blue}{#2}}\end{flushleft}}
\newcommand{\tm}[2][]{\begin{flushleft}\textbf{\encircle{Th\(^\mathbf{#1}\)} \textcolor{Red}{#2}}\end{flushleft}}
\newcommand{\inlineperm}[3][i]{{#1}_{#2}\dotsb{#1}_{#3}}
\usepackage{calc}
\definecolor{MMagenta}{rgb}{1.0, 0.0, 1.0}
\newcommand{\oversymbol}[2]{\overset{\textcolor{MMagenta}{\textbf{\textit{#1}}}}{\resizebox{\widthof{#1}}{\heightof{\(#2\)}}{\(#2\)}}}
\newcommand{\oversymbolnoref}[3][]{\underset{#1}{\overset{#2}{\resizebox{\widthof{\(#2\)}}{\heightof{\(#3\)}}{\(#3\)}}}}

\newenvironment{proof}{\paragraph{\(\square\)}}{\hfill\(\blacksquare\)}
\newenvironment{proof*}{\paragraph{\(\overset{\Rightarrow}{\square}\)}}{\hfill\(\blacksquare\)}
\newenvironment{proof**}{\paragraph{\(\overset{\Leftarrow}{\square}\)}}{\hfill\(\blacksquare\)}
%--------------------------------------------------------------------------------------------------------------------------
\let\oldexists\exists
\renewcommand{\exists}{\oldexists\,}
\newcommand{\existsone}{\ensuremath{\oldexists!\,}}
\let\oldforall\forall
\renewcommand{\forall}{\oldforall\,}
\let\oldnexists\nexists
\renewcommand{\nexists}{\oldnexists\,}
%--------------------------------------------------------------------------------------------------------------------------
\DeclareSymbolFont{extraup}          {U}{zavm}{m}{n}
\DeclareMathSymbol{\skull}{\mathalpha}{extraup}{119}
%--------------------------------------------------------------------------------------------------------------------------
\makeatletter
\renewcommand*\env@matrix[1][*\c@MaxMatrixCols c]{%
	\hskip -\arraycolsep
	\let\@ifnextchar\new@ifnextchar
	\array{#1}}
\makeatother
\DeclareMathOperator{\rank}{rank}
%--------------------------------------------------------------------------------------------------------------------------
\makeatletter
\newcommand{\hcdotsfor}[1]{%
	\ifx[#1\@xp\shcdots@for\else\hcdots@for\@ne{#1}\fi}
\def\shcdots@for#1]{\hcdots@for{#1}}
\def\hcdots@for#1#2{\multicolumn{#2}c%
	{\m@th\dotsspace@1.5mu\mkern-#1\dotsspace@
		\xleaders\hbox{$\m@th\mkern#1\dotsspace@\cdot\mkern#1\dotsspace@$}%
		\hfill
		\mkern-#1\dotsspace@}%
}
\makeatother
%--------------------------------------------------------------------------------------------------------------------------
\makeatletter
\DeclareRobustCommand
\myvdots{\vbox{\baselineskip4\p@ \lineskiplimit\z@
		\hbox{.}\hbox{.}\hbox{.}}}
\makeatother
%--------------------------------------------------------------------------------------------------------------------------
\makeatletter
\let\save@mathaccent\mathaccent
\newcommand*\if@single[3]{%
	\setbox0\hbox{${\mathaccent"0362{#1}}^H$}%
	\setbox2\hbox{${\mathaccent"0362{\kern0pt#1}}^H$}%
	\ifdim\ht0=\ht2 #3\else #2\fi
}
%The bar will be moved to the right by a half of \macc@kerna, which is computed by amsmath:
\newcommand*\rel@kern[1]{\kern#1\dimexpr\macc@kerna}
%If there's a superscript following the bar, then no negative kern may follow the bar;
%an additional {} makes sure that the superscript is high enough in this case:
\newcommand*\augm[1]{\@ifnextchar^{{\wide@bar{#1}{0}}}{\wide@bar{#1}{1}}}
%Use a separate algorithm for single symbols:
\newcommand*\wide@bar[2]{\if@single{#1}{\wide@bar@{#1}{#2}{1}}{\wide@bar@{#1}{#2}{2}}}
\newcommand*\wide@bar@[3]{%
	\begingroup
	\def\mathaccent##1##2{%
		%Enable nesting of accents:
		\let\mathaccent\save@mathaccent
		%If there's more than a single symbol, use the first character instead (see below):
		\if#32 \let\macc@nucleus\first@char \fi
		%Determine the italic correction:
		\setbox\z@\hbox{$\macc@style{\macc@nucleus}_{}$}%
		\setbox\tw@\hbox{$\macc@style{\macc@nucleus}{}_{}$}%
		\dimen@\wd\tw@
		\advance\dimen@-\wd\z@
		%Now \dimen@ is the italic correction of the symbol.
		\divide\dimen@ 3
		\@tempdima\wd\tw@
		\advance\@tempdima-\scriptspace
		%Now \@tempdima is the width of the symbol.
		\divide\@tempdima 10
		\advance\dimen@-\@tempdima
		%Now \dimen@ = (italic correction / 3) - (Breite / 10)
		\ifdim\dimen@>\z@ \dimen@0pt\fi
		%The bar will be shortened in the case \dimen@<0 !
		\rel@kern{0.6}\kern-\dimen@
		\if#31
		\overline{\rel@kern{-0.6}\kern\dimen@\macc@nucleus\rel@kern{0.4}\kern\dimen@}%
		\advance\dimen@0.4\dimexpr\macc@kerna
		%Place the combined final kern (-\dimen@) if it is >0 or if a superscript follows:
		\let\final@kern#2%
		\ifdim\dimen@<\z@ \let\final@kern1\fi
		\if\final@kern1 \kern-\dimen@\fi
		\else
		\overline{\rel@kern{-0.6}\kern\dimen@#1}%
		\fi
	}%
	\macc@depth\@ne
	\let\math@bgroup\@empty \let\math@egroup\macc@set@skewchar
	\mathsurround\z@ \frozen@everymath{\mathgroup\macc@group\relax}%
	\macc@set@skewchar\relax
	\let\mathaccentV\macc@nested@a
	%The following initialises \macc@kerna and calls \mathaccent:
	\if#31
	\macc@nested@a\relax111{#1}%
	\else
	%If the argument consists of more than one symbol, and if the first token is
	%a letter, use that letter for the computations:
	\def\gobble@till@marker##1\endmarker{}%
	\futurelet\first@char\gobble@till@marker#1\endmarker
	\ifcat\noexpand\first@char A\else
	\def\first@char{}%
	\fi
	\macc@nested@a\relax111{\first@char}%
	\fi
	\endgroup
}
\makeatother
%--------------------------------------------------------------------------------------------------------------------------
\setlength{\parskip}{0em}
\setlength{\parindent}{1.25cm} %ГОСТ 7.32
\binoppenalty=\maxdimen
\relpenalty=\maxdimen
\begin{document}
\setlength{\jot}{.5\baselineskip}
	\title{
		\centering \vfill
		\vspace*{0.5cm}
		\textbf{\huge Линейная алгебра} \\\bigskip
		\large Конспект лекций Баскакова А.В.\\\smallskip
		1 --- 2 семестры\\
		\vspace*{2.5cm}
		\includegraphics[keepaspectratio=true, scale=0.75]{mephi_logo.pdf}\vfill
	}
	\author{НИЯУ \textquote{МИФИ}}
	\date{2020 г.}
	
	\maketitle
	
	\begin{abstract}
		
		Данное пособие предназначено для студентов первого курса (в основном второго семестра) \textbf{ИЯФиТ}.
		
		Конспект представляет собой ответы на экзаменационные вопросы (касающиеся линейной алгебры) по курсу лекций \textbf{Баскакова Алексея Викторовича}, которые были предложены студентам в 2019--2020-х годах при подготовке к сдаче экзаменов по дисциплинам \enquote{\emph{Аналитическая геометрия}} и \enquote{\emph{Линейная алгебра}} в первом и втором семестрах соответственно.
		
		Хочется особенно подчеркнуть, что данный сборник предназначен исключительно для ревизии вопросов линейной алгебры первого курса нашего университета, поскольку автор не ставит своей задачей превзойти лекции по содержанию или же по форме изложения материала.
		
		Таким образом, \textbf{пособие не может заменить ваши собственные конспекты лекций}.
		
		Сам же курс лектора, по его собственным словам, основывается на учебнике \textquote{Линейная алгебра и некоторые её приложения}, 1985 г. под авторством Л.И.~Головиной.
	\end{abstract}

	\nomenclature[01]{\textbf{\underline{Опр}}}{Определение}
	\nomenclature[02]{\textbf{\underline{Пр}}}{Пример}
	\nomenclature[03]{\textbf{\ensq{Л}}}{Лемма}
	\nomenclature[04]{\textbf{\encircle{Th}}}{Теорема}
	\nomenclature[05]{$\square$}{Начало доказательства}
	\nomenclature[06]{$\blacksquare$}{Конец доказательства}
	\nomenclature[07]{$\mathbb{N}$}{Множество натуральных чисел}
	\nomenclature[08]{$\mathbb{Z}$}{Множество целых чисел}
	\nomenclature[09]{$\mathbb{R}$}{Множество действительных чисел}
	\nomenclature[10]{$\forall$}{Квантор всеобщности (любой, все)}
	\nomenclature[11]{$\exists$}{Квантор существования (существует)}
	\nomenclature[12]{$\nexists$}{Не существует}
	\nomenclature[13]{$\existsone$}{Существует, причём единственный}
	\nomenclature[14]{$\in$}{Элемент принадлежит множеству}
	\nomenclature[15]{$\subset$}{Множество содержится во множестве}
	\nomenclature[16]{$\cup$}{Объединение множеств}
	\nomenclature[17]{$\cap$}{Пересечение множеств}
	\nomenclature[18]{$\displaystyle\sum_{\mathfrak{A}}^{\mathfrak{B}}$}{Сумма по элементам $\mathfrak{AB}$}
	\nomenclature[19]{$\displaystyle\prod_{\mathfrak{A}}^{\mathfrak{B}}$}{Произведение по элементам $\mathfrak{AB}$}
	\nomenclature[20]{$\overline{a,b}$}{Целые числа на отрезке $\left[a;b\right]$}
	\nomenclature[21]{$\skull$}{Противоречие}
	\nomenclature[22]{$\left\langle\dotsb\right\rangle$}{Комментарий}
	
	\tableofcontents
	\printnomenclature
	
	% cd /d d:\texstudio-projects\
	% makeindex "Linear Algebra notes.nlo" -s nomencl.ist -o "Linear Algebra notes.nls"
	
	\chapter{Перестановки}
	\section{Чётность перестановки}
	\subsection{Перестановка}
	\df[Перестановка]
	
	Расставим числа $1,2,3,\dots ,n$ в каком-то порядке, тогда (для $n=5$):
		\begin{align*}
		&\begin{pmatrix}3&4&1&5&2\end{pmatrix}\text{ --- \underline{перестановка}}\\		
		&\begin{pmatrix}1&2&3&4&5\end{pmatrix}\text{ --- \underline{единичная перестановка}}
		\end{align*}
		
	Для $n$ элементов существует $n!$ перестановок 
	\[\begin{pmatrix}i_1&i_2&i_3&\dotsb&i_n\end{pmatrix}\]
	\subsection{Чётность}
	\df[Инверсия]
	
	В перестановке $\begin{pmatrix} i_1&i_2&\cdots&i_n \end{pmatrix}$ элементы $i_k$ и $i_p$ образуют инверсию, если $k<p$, но $i_k>i_p$
	\df[Чётность]
	
	Чётностью перестановки называется чётность числа инверсий.
	\section{Изменение чётности перестановки при транспозиции}
	\subsection{Транспозиции}
	\df[Транспозиция]
	
	Транспозицией перестановки называется перемена местами любых двух элементов перестановки.
	\df[Элементарная транспозиция (ЭТ)]
	
	Перемена местами двух соседних элементов перестановки --- элементарная транспозиция (ЭТ).
	
	\lm{При элементарной транспозиции чётность перестановки меняется}
	
	\begin{proof}
		
	\[
	\begin{gathered}
	\begin{pmatrix}i_1&i_2&i_3&\cdots&i_k&i_{k+1}&\cdots&i_n\end{pmatrix}\\
	\big\Downarrow\\
	\begin{pmatrix}i_1&i_2&i_3&\cdots&i_{k+1}&i_{k}&\cdots&i_n\end{pmatrix}
	\end{gathered}
	\]
	
	Инверсии, которые $i_k$ и $i_{k+1}$ составляли с остальными элементами, сохранились. Инверсия, связанная с перестановкой $i_k$ и $i_{k+1}$ либо появилась, либо исчезла. 
	
	Таким образом, количество инверсий изменилось на $1$, следовательно, чётность перестановки изменилась. \end{proof}
	\subsection{Изменение чётности}
	\tm{При любой транспозиции чётность перестановки меняется}
	
	\begin{proof}
	\[
	\begin{pmatrix} i_1&i_2&\cdots&\tikzmark{a}{i_k}&i_{k+1}&\cdots&i_{l-1}&\tikzmark{b}{i_l}&\cdots&i_n\end{pmatrix}
	\begin{tikzpicture}[overlay, remember picture]
		\draw[LaTeX-LaTeX] (a.north east) to [out=20, in=160](b.north west);
	\end{tikzpicture}
	\]
	
	Переставим элемент $i_k$ со впереди стоящим элементом вплоть до места с номером $l$ (всего $[l-k]$ ЭТ).
	
	Элемент $i_l$ оказался на $(l-1)$-ом месте. Перемещаем его элементарными транспозициями на $k$-ое место (всего $[l-1-k]$ ЭТ).
	
	Свели транспозицию к $\left[(l-k)+(l-1-k)\right] = \left[2(l-k)-1\right]$ --- нечётному числу~ЭТ $\implies$ сменили чётность нечётное число раз $\implies$ чётность изменилась. \end{proof}
	\section{Обратная перестановка}
	\subsection{Обратная перестановка}
	\df[Обратная перестановка]
	
	Пусть переставили элементы $\left\{1;2;3;\dotsb;n\right\}$:
	\[\begin{pmatrix} 1\\i_1\end{pmatrix}\begin{pmatrix} 2\\i_2\end{pmatrix}\dotsb\begin{pmatrix} n\\i_n\end{pmatrix}\]
	
	Переставим столбцы так, чтобы нижняя строка превратилась в единичную перестановку:
	\[\begin{pmatrix} j_1\\1\end{pmatrix}\begin{pmatrix} j_2\\2\end{pmatrix}\cdots\begin{pmatrix} j_n\\n\end{pmatrix}\]
	
	Получим перестановку $\begin{pmatrix} j_1&j_2&\cdots&j_n\end{pmatrix}$, называемую обратной к перестановке $\begin{pmatrix} i_1&i_2&\cdots&i_n\end{pmatrix}$.
	\subsection{Чётность обратной перестановки}
	\tm{Чётность прямой и обратной перестановок совпадает}
	
	\begin{proof} При перестановке столбцов совершается одинаковое число транспозиций над верхней и нижней перестановками. \end{proof}
	\chapter{Определитель матрицы}
	\section{Определение определителя матрицы}
	\df[Определитель матрицы]
	
	Для матрицы $M$ размером $n\times n$: 
	\[
	|M|\equiv \det M = \sum_{\left(i_1\,\inlineperm{2}{n}\right)} \left(-1\right)^{\upsigma\left(\inlineperm{1}{n}\right)} a_{1i_1}a_{2i_2}\dotsm a_{ni_n},
	\]
	
	где $\upsigma(\inlineperm{1}{n})$ --- дефект перестановки $\begin{pmatrix} i_1&i_2&\cdots&i_n\end{pmatrix}$, численно равный количеству инверсий этой перестановки.
	\ex
	
	Пусть дана матрица $\underset{2\times 2}{M}=\begin{pmatrix} a_{11}&a_{12}\\a_{21}&a_{22}\end{pmatrix}$. 
	
	Тогда её определитель $\displaystyle \det\underset{2\times 2}{M}=\sum_{\substack{(1\,\,2)\\(2\,\,1)}}(-1)^{\upsigma(i_1\,\,i_2)}a_{1i_1}a_{2i_2}=(-1)^{\upsigma(1\,\,2)}a_{11}a_{22}+{}\\{}+(-1)^{\upsigma(2\,\,1)}a_{12}a_{21}=a_{11}a_{22}-a_{12}a_{21}$.
	\section{Теорема об определителе транспонированной матрицы}
	\subsection{Транспонированная матрица}
	\df[Транспонированная матрица]
	
	Матрица $\underset{n\times n}{B}$ является транспонированной матрицей $\underset{n\times n}{A}$:
	\[
	\underset{n\times n}{B} = \underset{n\times n}{A^{\mathrm{T}}}\text{\,, если} \left(\forall i\leq j,\,j\leq n\right) b_{ij}=a_{ji}
	\]
	\ex
	
	$A=\begin{pmatrix} a_{11}&a_{12}\\a_{21}&a_{22}\end{pmatrix}\implies A^{\mathrm{T}}=\begin{pmatrix} a_{11}&a_{21}\\a_{12}&a_{22}\end{pmatrix}$
	\subsection{Теорема об определителе транспонированной матрицы}
	\tm{Если для матрицы $\boldsymbol{A}$ матрица $\boldsymbol{B=A^{\mathrm{T}}}$, то $\boldsymbol{|B|=|A|}$, т.е. $\boldsymbol{\left|A^{\mathrm{T}}\right|=|A|}$}
	
	\begin{proof}
	\[
	|B| = \sum_{(\inlineperm{1}{n})}(-1)^{\upsigma(\inlineperm{1}{n})}b_{1i_1}b_{2i_2}\dotsm b_{ni_n}=\sum_{(\inlineperm{1}{n})}(-1)^{\upsigma(\inlineperm{1}{n})}a_{i_1 1}a_{i_2 2}\dotsm a_{i_n n}=\dotso
	\]
	
	Выставляя элементы $a_{i_k k}$ в порядке возрастания первых индексов, получим перестановку $\begin{pmatrix} j_1&j_2&\cdots&j_n\end{pmatrix}$, обратную к $\begin{pmatrix} i_1&i_2&\cdots&i_n\end{pmatrix}$. Причём, $\upsigma(\inlineperm[j]{1}{n}) = \upsigma(\inlineperm{1}{n})$
	
	\[
	\dotso = \sum_{(\inlineperm[j]{1}{n})}(-1)^{\upsigma(\inlineperm[j]{1}{n})}a_{1j_1}a_{2j_2}\dotsm a_{n j_n} = |A|
	\]
	
	Таким образом, $\left|A^{\mathrm{T}}\right| = |A|$\end{proof}
	\section{Влияние элементарных преобразований на определитель матрицы}\label{2.3}
	\subsection{Умножение строки/столбца матрицы на число}
	\lm{Если для матриц $\boldsymbol{A\left\{a_{ij}\right\}_{i,j=1}^{n}}$ и $\boldsymbol{B\left\{b_{ij}\right\}_{i,j=1}^{n}}$ на фиксированной строке $\boldsymbol{l}$: $\boldsymbol{{\begin{cases}
	a_{lj}=\uplambda b_{lj},&j=\overline{1,n}
	\medskip\\a_{ij}=b_{ij},&i\neq l,\,j=\overline{1,n}\end{cases}}}$, то $\boldsymbol{|A|=\uplambda \left|B\right|}$}
	
	\begin{proof}
	\begin{align*} 
	|A|&=\sum_{(\inlineperm{1}{n})}(-1)^{\upsigma(\inlineperm{1}{n})}a_{1i_1}\dotsm a_{li_l}\dotsm a_{ni_n}=
	\\&=\sum_{(\inlineperm{1}{n})}(-1)^{\upsigma(\inlineperm{1}{n})}a_{1i_1}\dotsm \uplambda b_{li_l}\dotsm a_{ni_n}=
	\\&=\uplambda\cdot\smashoperator[l]{\sum_{(\inlineperm{1}{n})}}(-1)^{\upsigma(\inlineperm{1}{n})}b_{1i_1}\dotsm b_{li_l}\dotsm b_{ni_n}=
	\\&=\uplambda \left|B\right|
	\end{align*}\end{proof}
	
	Поскольку $|A|=\left|A^{\mathrm{T}}\right|$, то свойство верно и для столбцов.
	
	\subsection{Перестановка строки/столбца матрицы}
	\lm{Если для матриц $\boldsymbol{A\left\{a_{ij}\right\}_{i,j=1}^{n}}$ и $\boldsymbol{B\left\{b_{ij}\right\}_{i,j=1}^{n}}$ на фиксированных строках $\boldsymbol{k}$ и $\boldsymbol{l}$: $\boldsymbol{{b_{ij}=\begin{cases}
	a_{ij},&i\neq k\text{ и }i\neq l
	\\a_{kj},&\text{строка }l
	\\a_{lj},&\text{строка }k
	\end{cases}}}$, то $\boldsymbol{|A|=-|B|}$}
	
	\begin{proof}
	\begin{align*}
	|B|&=\sum_{(\inlineperm{1}{n})}(-1)^{\upsigma(\inlineperm{1}{l}\,\inlineperm{k}{n})}b_{1i_1}\cdots b_{li_l}b_{ki_k}\cdots b_{ni_n}=
	\\&=\sum_{(\inlineperm{1}{n})}(-1)^{\upsigma(\inlineperm{1}{l}\,\inlineperm{k}{n})}a_{1i_1}\cdots a_{ki_l}a_{li_k}\cdots a_{ni_n}
	\\[\baselineskip]
	|A|&=\sum_{(\inlineperm{1}{n})}(-1)^{\upsigma(\inlineperm{1}{k}\,\inlineperm{l}{n})}a_{1i_1}\cdots a_{ki_l}a_{li_k}\cdots a_{ni_n}=
	\\&=\sum_{(\inlineperm{1}{n})}(-1)\cdot (-1)^{\upsigma(\inlineperm{1}{l}\,\inlineperm{k}{n})}a_{1i_1}\cdots a_{ki_l}a_{li_k}\cdots a_{ni_n}=
	\\&=-|B|
	\end{align*}\end{proof}
	
	Поскольку $|A|=\left|A^{\mathrm{T}}\right|$, то свойство верно и для столбцов.
	
	\subsection{Прибавление строк/столбцов другой матрицы}
	\lm{Если для матриц $\boldsymbol{A\left\{a_{ij}\right\}_{i,j=1}^{n}}$, $\boldsymbol{B\left\{b_{ij}\right\}_{i,j=1}^{n}}$, $\boldsymbol{C\left\{c_{ij}\right\}_{i,j=1}^{n}}$ на фиксированной\\строке $\boldsymbol{l}$:
	\smallskip\\$\boldsymbol{{a_{ij}=\begin{cases}
	b_{ij}=c_{ij},&i\neq l,\,j=\overline{1,n}
	\smallskip\\b_{lj}+c_{lj},&\text{строка }l,\,j=\overline{1,n}
	\end{cases}}}$, то $\boldsymbol{|A|=|B|+|C|}$}\label{2.3.3 L}
	
	\begin{proof}
	\begin{align*}
	|A| &= \sum_{(\inlineperm{1}{n})}(-1)^{\upsigma(\inlineperm{1}{n})}a_{1i_1}\cdots a_{li_l}\cdots a_{ni_n}=
	\\&=\sum_{(\inlineperm{1}{n})}(-1)^{\upsigma(\inlineperm{1}{n})}a_{1i_1}\cdots (b_{li_l}+c_{li_l})\cdots a_{ni_n}=
	\\&=\sum_{(\inlineperm{1}{n})}(-1)^{\upsigma(\inlineperm{1}{n})}b_{1i_1}\cdots b_{li_l}\cdots b_{ni_n} + \sum_{(\inlineperm{1}{n})}(-1)^{\upsigma(\inlineperm{1}{n})}c_{1i_1}\cdots c_{li_l}\cdots c_{ni_n}=
	\\&=|B|+|C|
	\end{align*}\end{proof}
	
	Поскольку $|A|=\left|A^{\mathrm{T}}\right|$, то свойство верно и для столбцов.
	
	\section{Вычисление определителя треугольной матрицы}
	\subsection{Треугольный вид матрицы}
	\df[Верхнетреугольная матрица]
	
	 Матрица $A\left\{a_{ij}\right\}_{i,j=1}^{n}$ имеет верхнетреугольный вид, если\\\({\left(\forall j =\overline{1,n},\, \forall i =\overline{2,n}\colon i>j\right) a_{ij} = 0}\).
	\[
	A=\begin{pmatrix} 
	a_{11}&a_{12}&\dotsb&a_{1(n-1)}&a_{1n}\\
	0&a_{22}&\cdots&a_{2(n-1)}&a_{2n}\\
	0&0&\cdots&a_{3(n-1)}&a_{3n}\\
	\vdots&\vdots&\ddots&\vdots&\vdots\\
	0&0&\cdots&0&a_{nn}
	\end{pmatrix}
	\]
	\lm{Определитель верхнетреугольной матрицы равен $\displaystyle\boldsymbol{|A|=\prod_{i=1}^{n}a_{ii}}$}
	\begin{proof}
	\[
	|A| = \sum_{(\inlineperm{1}{n})}(-1)^{\upsigma(\inlineperm{1}{n})}a_{1i_1}a_{2i_2}\cdots a_{ni_n}=\dotso
	\]
	
	Заметим, что любая не единичная перестановка будет содержать нуль:
	\[
	\dotso=a_{11}a_{22}\cdots a_{nn}=\prod_{i=1}^{n}a_{ii}
	\]\end{proof}

	\subsection{Приведение матрицы к верхнетреугольному виду}
	\lm{Любую матрицу можно привести к верхнетреугольному виду путём элементарных преобразований строк/столбцов}
	
	\begin{proof}
	Применим метод математической индукции:
	\begin{enumerate}[1)]
		\item $n=1$:
		\begin{enumerate}[(1.1)]
			\item $\left|a_{11}\right|=a_{11}$ --- верно
		\end{enumerate}
		\item Пусть лемма верна для $n=m$
		\item Докажем её для $n=m+1$:
		\begin{enumerate}[(3.1)]
			\item Если $\left(\forall i =\overline{1,m+1},\,\forall j=\overline{1,m+1}\right) a_{ij}=0$ --- уже верхнетреугольный вид
			\item Если $\exists a_{i_{0}j_{0}}\neq 0$\,:
				\begin{itemize}
					\item Переставим строку $i_0$ с первой строкой, а столбец $j_0$ с первым столбцом
					\[
					\begin{pmatrix} 
					a_{i_0 j_0}&\cdots&a_{1(m+1)}^{\star}\\
					a_{21}^{\star}&\cdots&a_{2(m+1)}^{\star}\\
					\vdots&\ddots&\vdots\\
					a_{\substack{(m+1)\\1}}^{\star}&\cdots&a_{\substack{(m+1)\\(m+1)}}^{\star}
					\end{pmatrix}
					\]
					\item Занулим элементы, стоящие под $a_{i_0 j_0}$, поочерёдно вычтя из второй по $(m+1)$--ю строки первую, умноженную на соответственно: $\dfrac{a_{21}^{\star}}{a_{i_0 j_0}}$,~$\dfrac{a_{31}^{\star}}{a_{i_0 j_0}}$,~$\dotsc$,~$\dfrac{a_{(m+1)1}^{\star}}{a_{i_0 j_0}}$
					\[
					\begin{pmatrix}
						a_{i_0 j_0} &\cdots& a_{1(m+1)}^{\star}\\
						0 &\tikz[overlay, remember picture, baseline=(Od.base)]\node (Od){\(\phantom{\cdots}\)};&\phantom{a_{1(m+1)}^{\star}}\\
						0 &\phantom{\cdots}&\phantom{a_{1(m+1)}^{\star}}\\
						\vdots &\phantom{\cdots}&\phantom{a_{1(m+1)}^{\star}}\\
						0 &\phantom{\cdots}&\tikz[overlay, remember picture, baseline=(O.base)]\node (O) {\(\phantom{0_{m+1}}\)};
					\end{pmatrix}
					\begin{tikzpicture}[overlay, remember picture]
						\draw[fill=green!20, thick, rounded corners] (Od.north west) rectangle (O.east) node [midway, align=center] {Матрица\\размером\\$m\times m$};
					\end{tikzpicture}
					\]
					\item По предположению индукции можем привести матрицу размером $m\times m$ к верхнетреугольному виду.
				\end{itemize}
		\end{enumerate}
	\end{enumerate}\end{proof}
	\section{Леммы о разложении определителя по последней строке}
	\subsection{Дополнительный минор, алгебраическое дополнение}
	\df[Дополнительный минор]\label{2.5.1}
	
	В квадратной матрице $A\left\{a_{ij}\right\}_{i,j=1}^{n}$ вычеркнем строку $k$, столбец $l$. Определитель полученной матрицы называется дополнительным минором $\mathrm{M}_{kl}$.
	
	\ex
	
	\(\begin{aligned}\displaystyle 
	&\underset{3\times 3}{A}=\begin{pmatrix}
		a_{11}&a_{12}&a_{13}\\
		a_{21}&a_{22}&a_{23}\\
		a_{31}&a_{32}&a_{33}\\
	\end{pmatrix}\\
	&\mathrm{M}_{22}\Big(\underset{3\times 3}{A}\Big)=
	\begin{vmatrix}
		a_{11}&\tikzmarkempty{A}{a_{12}}&a_{13}\\
		\tikzmarkempty{C}{a_{21}}&a_{22}&\tikzmarkempty{D}{a_{23}}\\
		a_{31}&\tikzmarkempty{B}{a_{32}}&a_{33}\\
	\end{vmatrix}=
	\begin{vmatrix}
		a_{11}&a_{13}\\
		a_{31}&a_{33}\\
	\end{vmatrix}=a_{11}a_{33}-a_{31}a_{13}
	\begin{tikzpicture}[overlay, remember picture]
		\draw[red, very thick] (A.north) to (B.south);
		\draw[red, very thick] (C.west) to (D.east);
	\end{tikzpicture}\end{aligned}\)
	
	\df[Алгебраическое дополнение]
	
	Алгебраическое дополнение, соответствующее элементу $a_{kl}$, равно\\$\mathrm{A}_{kl}=(-1)^{k+l}\mathrm{M}_{kl}$
	
	\subsection{Лемма I}\label{2.5.2}
	
	\lm{Если в матрице $\boldsymbol{A\left\{a_{ij}\right\}_{i,j=1}^{n}}$, $\boldsymbol{\left(\forall j=\overline{1,n-1}\right)a_{nj}=0}$, то $\boldsymbol{|A|=a_{nn}\mathrm{A}_{nn}}$}
	\begin{proof}
	\begin{gather}
	\begin{align*}
		A &=\begin{pmatrix}
			a_{11}&a_{12}&\cdots&a_{1(n-1)}&a_{1n}\\
			a_{21}&a_{22}&\cdots&a_{2(n-1)}&a_{2n}\\
			\vdots&\vdots&\ddots&\vdots&\vdots\\
			a_{\substack{(n-1)\\1}}&a_{\substack{(n-1)\\2}}&\cdots&a_{\substack{(n-1)\\(n-1)}}&a_{\substack{(n-1)\\n}}\\
			0&0&\cdots&0&a_{nn}\\
			\end{pmatrix}
		\\[\baselineskip]
		|A| &= \sum_{(\inlineperm{1}{n-1}\,\,i_n)}(-1)^{\upsigma({\inlineperm{1}{n-1}\,\, i_n})}a_{1i_1}a_{2i_2}\cdots a_{ni_n}=\\
		&=0+0+0+\dotsb+0+\smashoperator[l]{\sum_{(\inlineperm{1}{n-1}\,\,n)}}(-1)^{\upsigma(\inlineperm{1}{n-1}\,\,n)}a_{1i_1}a_{2i_2}\cdots a_{\substack{(n-1)\\i_{n-1}}}a_{nn}=\\
		&=a_{nn}\cdot\smashoperator[l]{\sum_{(\inlineperm{1}{n-1})}}(-1)^{\upsigma(\inlineperm{1}{n-1})}a_{1i_1}a_{2i_2}\cdots a_{\substack{(n-1)\\i_{n-1}}}=\\
		&=a_{nn}\mathrm{M}_{nn}=a_{nn}\cdot (-1)^{n+n}\mathrm{M}_{nn}=a_{nn}\mathrm{A}_{nn}
	\end{align*}
	\end{gather}\end{proof}
	\subsection{Лемма II}\label{2.5.3}
	
	\lm{Если в матрице \(\boldsymbol{A\left\{a_{ij}\right\}_{i,j=1}^{n}}\), \(\boldsymbol{\left(\forall j\neq l\right) a_{nj}=0}\), то \(\boldsymbol{|A|=a_{nl}\mathrm{A}_{nl}}\)}
	\begin{proof}
	\[
		A = \begin{pmatrix}
				a_{11}&\cdots&a_{1l}&\cdots&a_{1n}\\
				\vdots&\vdots&\ddots&\vdots&\vdots\\
				0&\cdots&a_{nl}&\cdots&0\\
			\end{pmatrix}
	\]
	
	Переставим столбец \(l\) на место столбца \(n\) путём последовательных элементарных транспозиций (всего \([n-l]\) ЭТ):
	
	\begin{align*}
		|A|&=(-1)^{n-l}\cdot\begin{vmatrix}
								a_{11}&\cdots&a_{1(l-1)}&a_{1(l+1)}&\cdots&a_{1n}&a_{1l}\\
								\vdots&\vdots&\vdots&\ddots&\vdots&\vdots&\vdots\\
								0&\cdots&0&0&\cdots&0&a_{nl}\\
							\end{vmatrix}
		\oversymbol{\nameref{2.5.2}}{=}(-1)^{n-l}\cdot a_{nl}\mathrm{M}_{nl}=\\
		&=\Big\langle(-1)^{n-l}=(-1)^{n+l}\Big\rangle=a_{nl}\cdot (-1)^{n+l}\mathrm{M}_{nl}=a_{nl}\mathrm{A}_{nl}
	\end{align*}\end{proof}
	
	\subsection{Лемма III}\label{2.5.4}
	
	\lm{Определитель матрицы \(\boldsymbol{A\left\{a_{ij}\right\}_{i,j=1}^{n}}\) равен \(\displaystyle\boldsymbol{ |A|=\sum_{i=1}^{n}a_{ni}\mathrm{A}_{ni}}\)}
	\begin{proof}
	\[
		A=\begin{pmatrix}a_{11}&\cdots&a_{1n}\\
						\vdots&\ddots&\vdots\\
						a_{n1}&\cdots&a_{nn}\\
		\end{pmatrix}
	\]
	
	По \hyperref[2.3.3 L]{\emph{\color{MMagenta}свойству прибавления строк другой матрицы}} можем представить определитель \(|A|\) как сумму определителей \(\left|A^{(i)}\right|\) матриц \(A^{(i)}\), для которых элементы последней строки равны \(a_{nj}^{(i)}=\begin{cases}a_{ni},&j=i\\0,&j\neq i\end{cases}\)\,, а остальные элементы совпадают с элементами матрицы \(A\):
	\[
		|A|=\underbrace{\begin{vmatrix} a_{11}&a_{12}&\hcdotsfor{2}&a_{1n} \\ \vdots&\vdots&\ddots&\vdots&\vdots \\ a_{n1}&0&\hcdotsfor{2}&0 \end{vmatrix}}_{\left|A^{(1)}\right|} + \underbrace{\begin{vmatrix} a_{11}&a_{12}&\hcdotsfor{2}&a_{1n} \\ \vdots&\vdots&\ddots&\vdots&\vdots \\ 0&a_{n2}&\hcdotsfor{2}&0 \end{vmatrix}}_{\left|A^{(2)}\right|}+\cdots+\underbrace{\begin{vmatrix} a_{11}&a_{12}&\hcdotsfor{2}&a_{1n} \\ \vdots&\vdots&\ddots&\vdots&\vdots \\ 0&0&\hcdotsfor{2}&a_{nn} \end{vmatrix}}_{\left|A^{(n)}\right|}
	\]
	
	\[
		|A|=\sum_{i=1}^{n}\left|A^{(i)}\right|\oversymbol{\nameref{2.5.3}}{=}\sum_{i=1}^{n}a_{ni}\mathrm{A}_{ni}
	\]\end{proof}
	
	\section{Разложение определителя по любой строке или столбцу}
	\tm{Для матрицы \(\boldsymbol{A\left\{a_{ij}\right\}_{i,j=1}^{n}}\) \(\displaystyle\boldsymbol{\left(\forall k=\overline{1,n}\right)|A|=\sum_{i=1}^{n}a_{ki}\mathrm{A}_{ki}}\)}
	\begin{proof}
	Начиная со строки \(k\), последовательно переставим строки вплоть до строки \(n\):
	
	\begin{align*}
		|A|&=(-1)^{n-k}\cdot\begin{vmatrix}
						a_{11}&\cdots&a_{1n}\\
						\vdots&\ddots&\vdots\\
						a_{\substack{(k-1)\\1}}&\cdots&a_{\substack{(k-1)\\n}}\\
						a_{\substack{(k+1)\\1}}&\cdots&a_{\substack{(k+1)\\n}}\\
						\vdots&\ddots&\vdots\\
						a_{n1}&\cdots&a_{nn}\\
						\tikz[overlay, remember picture, baseline=(Ak1.base)]\node (Ak1){\(a_{k1}\)};&\cdots&a_{kn}\\
						\end{vmatrix}\oversymbol{\nameref{2.5.4}}{=}(-1)^{n-k}\cdot\sum_{i=1}^{n}a_{ki}\mathrm{A}_{ki}=\\
			&=(-1)^{n-k}\sum_{i=1}^{n}(-1)^{n+i}a_{ki}\mathrm{M}_{ki}=\sum_{i=1}^{n}(-1)^{2n-k+i}a_{ki}\mathrm{M}_{ki}=\\
			&=\sum_{i=1}^{n}(-1)^{k+i}a_{ki}\mathrm{M}_{ki}=\sum_{i=1}^{n}a_{ki}\mathrm{A}_{ki}
		\begin{tikzpicture}[overlay, remember picture]
			\draw [LaTeX-, thick] (Ak1.west) ++(-.5em, 0) to ++(-2em, 0) node [align=right, left, fill=green!20, outer sep=.05em, inner sep=.1em, rounded corners] {\scriptsize\(\big\langle n\text{--я строка}\big\rangle\)};
		\end{tikzpicture}
	\end{align*}\end{proof}
	\section{Фальшивое разложение определителя}
	\subsection{Определитель матрицы с одинаковыми строками/столбцами}
	\lm{Если матрица содержит одинаковые строки/столбцы, то её определитель равен нулю}
	\begin{proof}
	\[A=\begin{pmatrix}a_{11}&\dotsb&a_{1n}\\
						\vdots&\ddots&\vdots\\
						C_{1}&\dotsb&C_{n}\\
						\vdots&\ddots&\vdots\\
						C_{1}&\dotsb&C_{n}\\
						\vdots&\ddots&\vdots\\
						a_{n1}&\dotsb&a_{nn}\\
						\end{pmatrix}\text{или } B = \begin{pmatrix}
														b_{11}&\dotsb&V_{1}&\dotsb&V_{1}&\dotsb&b_{1n}\\
														\vdots&\ddots&\vdots&\ddots&\vdots&\ddots&\vdots\\
														b_{n1}&\dotsb&V_{n}&\dotsb&V_{n}&\dotsb&b_{nn}\\
													\end{pmatrix}\]
													
	При вычислении определителя матриц переставим одинаковые строки/столбцы местами, получим:
	\[ |A| = (-1)\cdot|A|\implies |A| = 0\text{ (аналогично \(|B|=0\))}\]\end{proof}
	
	\subsection{Теорема о произведении элементов матрицы одной строки на алгебраические дополнения элементов другой строки}
	\tm{Если задана матрица \(\boldsymbol{A\left\{a_{ij}\right\}_{i,j=1}^{n}}\), то \(\displaystyle\boldsymbol{\left(\forall l \neq k \right)\sum_{j=1}^{n}a_{lj}\mathrm{A}_{kj}=0}\)}\label{2.7.2 Th}
	\begin{proof}
	Составим матрицу \(B\left\{b_{ij}\right\}_{i,j=1}^{n}\), в которой \(b_{ij}=\begin{cases}a_{ij},&i\neq k\\a_{lj},&i=k\end{cases}\)
	\[A=\begin{pmatrix}a_{11}&\dotsb&a_{1n}\\
						\vdots&\ddots&\vdots\\
						a_{l1}&\dotsb&a_{ln}\\
						\vdots&\ddots&\vdots\\
						a_{k1}&\dotsb&a_{kn}\\
						\vdots&\ddots&\vdots\\
						a_{1n}&\dotsb&a_{nn}\\\end{pmatrix},~ B=\begin{pmatrix}
																			b_{11}&\dotsb&b_{1n}\\
																			\vdots&\ddots&\vdots\\
																			b_{l1}&\dotsb&b_{ln}\\
																			\vdots&\ddots&\vdots\\
																			b_{k1}&\dotsb&b_{kn}\\
																			\vdots&\ddots&\vdots\\
																			b_{1n}&\dotsb&b_{nn}\\
																\end{pmatrix}=\begin{pmatrix}
																				a_{11}&\dotsb&a_{1n}\\
																				\vdots&\ddots&\vdots\\
																				a_{l1}&\dotsb&a_{ln}\\
																				\vdots&\ddots&\vdots\\
																				a_{l1}&\dotsb&a_{ln}\\
																				\vdots&\ddots&\vdots\\
																				a_{1n}&\dotsb&a_{nn}\\
																			\end{pmatrix}\]
																						
	Поскольку \(B\) содержит две одинаковые строки, то \(|B|=0\). С другой стороны, разложив \(|B|\) по \(k\text{--ой}\) строке, получим \(\displaystyle |B|=\sum_{j=1}^{n}b_{kj}\mathrm{B}_{kj}=\sum_{j=1}^{n}a_{lj}\mathrm{A}_{kj}=0\)\end{proof}
	\chapter{Произведение матриц}
	\section{Свойства произведения матриц}
	\subsection{Определение произведения матриц}
	\df[Произведение матриц]
	
	Если заданы матрицы \(\underset{m\times n}{A}\), \(\underset{n\times k}{B}\), то произведением \(A\cdot B\) называется матрица \(\underset{m\times k}{C}\) такая, что:
	
	\begin{align*}
		\underset{m\times n}{A}\cdot\underset{n\times k}{B} &= \underset{m\times k}{C}\\
													\left(\forall i=\overline{1,m}\,,\,\forall j=\overline{1,k}\right)
													c_{ij} &= \sum_{t=1}^{n}a_{it}b_{tj}
	\end{align*}
	\subsection{Дистрибутивность}
	\lm{Произведение матриц дистрибутивно}
	
	\[
		\underbrace{\underset{n\times k}{A}\bigg(\overbrace{\uplambda\underset{k\times m}{B}+\upmu\underset{k\times m}{C}}^{R\left\{r_{ij}\right\}}\bigg)}_{D\left\{d_{ij}\right\}} = \underbrace{\overbrace{\uplambda\underset{n\times m}{AB}}^{P\left\{p_{ij}\right\}}+\overbrace{\upmu\underset{n\times m}{AC}}^{Q\left\{q_{ij}\right\}}}_{F\left\{f_{ij}\right\}},\,\Big(\{\uplambda,\upmu\}\subset\mathbb{R}\Big)
	\]
	
	\begin{proof}
	\begin{align*}
		d_{ij}&=\sum_{t=1}^{k}a_{it}r_{tj}=\sum_{t=1}^{k}a_{it}\left(\uplambda b_{tj}+\upmu c_{tj}\right)=\uplambda\sum_{t=1}^{k}a_{it}b_{tj}+\upmu\sum_{t=1}^{k}a_{it}c_{tj}=p_{ij}+q_{ij}=f_{ij}\implies\\[5pt]
			&\implies D = F
	\end{align*}\end{proof}
	\newpage\subsection{Ассоциативность}
	\lm{Произведение матриц ассоциативно}
	
	\begin{align*}
		&\bigg(\overbrace{\underset{m\times k}{A}\cdot\underset{k\times l}{B}}^{F\left\{f_{ij}\right\}}\bigg)\cdot\underset{l\times n}{C} = \underset{m\times n}{T}\\[5pt]
		&\underset{m\times k}{A}\cdot\bigg(\overbrace{\underset{k\times l}{B}\cdot\underset{l\times n}{C}}^{G\left\{g_{ij}\right\}}\bigg) = \underset{m\times n}{D}
	\end{align*}
	
	Покажем, что \(T=D\):
	\begin{proof}
	\begin{align*}
		t_{ij}&=\sum_{p=1}^{l}f_{ip}c_{pj}=\sum_{p=1}^{l}c_{pj}\left(\sum_{q=1}^{k}a_{iq}b_{qp}\right)=\sum_{q=1}^{k}a_{iq}\left(\sum_{p=1}^{l}b_{qp}c_{pj}\right)=\sum_{q=1}^{k}a_{iq}g_{qj}=d_{ij}\implies\\[5pt]
			&\implies T = D
	\end{align*}\end{proof}
	\subsection{Некоммутативность (контрпример)}
	\lm{Произведение матриц некоммутативно}
	
	В общем случае \(A\cdot B\neq B\cdot A\)
	\ex
	
	\(\begin{aligned}&A=\begin{pmatrix}[rr]3&1\\-1&4\end{pmatrix},~B=\begin{pmatrix}[rr]2&1\\-3&0\end{pmatrix}\\
	&\begin{drcases}
	A\cdot B = \begin{pmatrix}[rr]3&1\\-1&4\end{pmatrix}\cdot\begin{pmatrix}[rr]2&1\\-3&0\end{pmatrix}=\begin{pmatrix}[rr]3&3\\-14&-1\end{pmatrix}\\
	B\cdot A=\begin{pmatrix}[rr]2&1\\-3&0\end{pmatrix}\cdot\begin{pmatrix}[rr]3&1\\-1&4\end{pmatrix}=\begin{pmatrix}[rr]5&6\\-9&-3\end{pmatrix}
	\end{drcases}\implies A\cdot B\neq B\cdot A\end{aligned}\)
	
	\section{Единичная матрица}
	\df[Единичная матрица]
	
	Единичная матрица --- матрица \(E\left\{e_{ij}\right\}_{i,j=1}^{n}\), \(\left(\forall i =\overline{1,n},\forall j =\overline{1,n}\right)\\e_{ij}=\tikzmark{K}{\updelta_{ij}}=\begin{cases}1,&i=j\\0,&i\neq j\end{cases}
		\begin{tikzpicture}[overlay, remember picture]
			\draw [LaTeX-, thick] (K.south) [out=-115, in=85] to ++(-1.5em, -2em) node [align=center, below] {\scriptsize\(\big\langle\)Символ Кронекера\(\big\rangle\)} ;
		\end{tikzpicture}
	\)
	
	\lm{Если \(\boldsymbol{\exists A\cdot E}\), то \(\boldsymbol{A\cdot E = A}\) (аналогично \(\boldsymbol{E\cdot B=B}\))}
	\begin{proof}
	\begin{align*}
		\underset{m\times n}{A}\cdot\underset{n\times n}{E}&=\underset{m\times n}{C}\\
														\left(\forall i=\overline{1,m},\forall j =\overline{1,n}\right)c_{ij}&=\sum_{t=1}^{n}a_{it}\updelta_{tj}=0+0+\dotsb+a_{ij}\updelta_{jj}=a_{ij}\implies C = A
	\end{align*}\end{proof}
	\section[Теорема об определителе произведения матриц \textit{(без доказательства)}]{Теорема об определителе произведения матриц \textit{(без доказательства)}\footnote[1]{Полное доказательство можно обнаружить в упомянутом в аннотации учебнике Л.И.~Головиной\\(Глава III. Линейные операторы / \S2. Действия над линейными операторами / Теорема 3).}}
	\tm{Если \(\boldsymbol{A}\) и \(\boldsymbol{B}\) --- квадратные матрицы одного порядка, то \(\boldsymbol{\left|A\cdot B\right|=|A|\cdot|B|}\)}
	\section{Обратная матрица}
	\subsection{Определение обратной матрицы}
	\df[Обратная матрица]
	
	Если для матрицы \(\underset{n\times n}{A}\) существует матрица \(\underset{n\times n}{B}\) такая, что \(A\cdot B = B\cdot A = E\), то матрицу \(B\) называют обратной к матрице \(A\) и обозначают \(B\equiv A^{-1}\).
	\subsection{Критерий существования обратной матрицы}
	\tm{Если у матрицы \(\boldsymbol{{\underset{n\times n}{A}}}\), \(\boldsymbol{{|A|\neq 0}}\), то \(\bm{{\existsone A^{-1}\colon A\cdot A^{-1} = A^{-1} \cdot A=E}}\)}
	
	\begin{proof}
		
		\begin{enumerate}[1)]
		\item Существование:
			\begin{itemize}
				\item Докажем, что если \(|A| = 0\), то \(\nexists A^{-1}\)\,:
			
			 Предположим противное --- \(A^{-1}\) существует и \(|A|=0\)\,:
			\begin{align*} 
				A\cdot A^{-1} &= E\\
				|A|\cdot\left|A^{-1}\right| &= |E|\\
				0 &= 1 \implies\skull
			\end{align*}
				\item Докажем, что если \(|A|\neq 0\), то \(\exists A^{-1}=\dfrac{1}{|A|}\big(A^{*}\big)^{\mathrm{T}}\), где \(A=\left\{a_{ij}\right\}_{i,j=1}^{n},\,A^{*}=\left\{\mathrm{A}_{ij}\right\}_{i,j=1}^{n}\).
			
			Рассмотрим \(B=\dfrac{1}{|A|}\big(A^{*}\big)^{\mathrm{T}}\) и докажем, что \(A\cdot B=B\cdot A=E\).
			
			\(\underset{n\times n}{A}\cdot\underset{n\times n}{B}=\underset{n\times n}{C}\), \(|A| = \Updelta\)
			\begin{align*}
				c_{ij}&=\sum_{t=1}^{n}a_{it}b_{tj}=\sum_{t=1}^{n}a_{it}\cdot\dfrac{1}{\Updelta}\mathrm{A}_{jt}=\dfrac{1}{\Updelta}\sum_{t=1}^{n}a_{it}\mathrm{A}_{jt}=\\
					&=\begin{cases}\dfrac{1}{\Updelta}\cdot\Updelta,&i=j\\0,&i\neq j\end{cases}=\begin{cases}1,&i=j\\0,&i\neq j\end{cases}\equiv\updelta_{ij}\implies\\
					&\implies C=E
			\end{align*}
			
			Аналогично \(B\cdot A=E\), т.к. разложение для определителя справедливо и для столбцов.
			\end{itemize}
		\item Единственность:
		
			 Пусть для матрицы \(A\) существуют матрицы \(B\) и \(C\) такие, что \(\begin{cases}A\cdot B=B\cdot A = E\\A\cdot C=C\cdot A= E\\ B\neq C\end{cases}\)
			 
			Тогда, \(\begin{rcases}C\cdot A\cdot B=\big(C\cdot A\big)\cdot B=E\cdot B=B\\C\cdot A\cdot B=C\cdot\big(A\cdot B\big)=C\cdot E= C\end{rcases}\implies B=C\implies\skull\)
		\end{enumerate}
	\end{proof}
	\subsection{Построение обратной матрицы через алгебраические дополнения}
	\ex
	
	\(A=\begin{pmatrix}1&2\\3&4\end{pmatrix}\)
	
	\begin{enumerate}[1)]
	\item \(\begin{alignedat}[t]{2}&\mathrm{A}_{11}=(-1)^{1+1}\mathrm{M}_{11}=4\hspace{25pt}&&\mathrm{A}_{12}=(-1)^{1+2}\mathrm{M}_{12}=-3\\&\mathrm{A}_{21}=(-1)^{2+1}\mathrm{M}_{21}=-2\hspace{25pt}&&\mathrm{A}_{22}=(-1)^{2+2}\mathrm{M}_{22}=1\end{alignedat}\)
	
	\item \(A^{*}=\begin{pmatrix}[rr]4&-3\\-2&1\end{pmatrix}\implies \big(A^{*}\big)^{\mathrm{T}}=\begin{pmatrix}[rr]4&-2\\-3&1\end{pmatrix}\)
	
	\item \(|A| = 4-6=-2\)
	
	\item \(A^{-1}=\dfrac{1}{|A|}\big(A^{*}\big)^{\mathrm{T}}=\begin{pmatrix}[rr]-2&1\\\tfrac{3}{2}&-\tfrac{1}{2}\end{pmatrix}\)
	
	\item \(\begin{alignedat}[t]{1}&A\cdot A^{-1} = \begin{pmatrix}1&2\\3&4\end{pmatrix}\cdot \begin{pmatrix}[rr]-2&1\\\tfrac{3}{2}&-\tfrac{1}{2}\end{pmatrix}=\begin{pmatrix}[rr]-2+3&1-1\\-6+6&3-2\end{pmatrix}=\begin{pmatrix}1&0\\0&1\end{pmatrix}=E\\&A^{-1}\cdot{A}=\begin{pmatrix}[rr]-2&1\\\tfrac{3}{2}&-\tfrac{1}{2}\end{pmatrix}\cdot \begin{pmatrix}1&2\\3&4\end{pmatrix}=\begin{pmatrix}[rr]-2+3&-4+4\\\tfrac{3}{2}-\tfrac{3}{2}&3-2\end{pmatrix}=\begin{pmatrix}1&0\\0&1\end{pmatrix}=E\end{alignedat}\)
	\end{enumerate}
	\subsection[Построение обратной матрицы методом ЭПС]{Построение обратной матрицы методом элементарных преобразований строк}
	\subsubsection{Умножение \(\boldsymbol{k}\)--ой строки на \(\boldsymbol{\upalpha\in\mathbb{R}}\)}
	\lm{Соответствует левостороннему домножению на \(\boldsymbol{{B\left\{b_{ij}\right\}_{i,j=1}^{n}}}\), \(\boldsymbol{{b_{ij}=\begin{cases}\upalpha,&i=k\text{ и }j=k\\\updelta_{ij},&i\neq k\text{ или }j\neq k\end{cases}}}\)}
	
	\begin{proof} \(B\cdot A=C\)
	\begin{enumerate}[{1.}1)]
		\item \(\big(i=k,\, j=\overline{1,n}\big)\)~\(\begin{aligned}[t]\displaystyle c_{kj} = \sum_{t=1}^{n}b_{kt}a_{tj}&=\begin{aligned}[t]\updelta_{k1}a_{1j}&+\updelta_{k2}a_{2j}+\dotsb+\updelta_{k(k-1)}a_{(k-1)j}+{}\\&{}+\upalpha a_{kj}+\dotsb+\updelta_{kn}a_{nj}=\end{aligned}\\&=0+0+\dotsb+0+\upalpha a_{kj}+\dotsb+0=\\&=\upalpha a_{kj}\end{aligned}\)
	
		\item \(\big(i\neq k,\, j=\overline{1,n}\big)\) \(\begin{aligned}[t]\displaystyle c_{ij}=\sum_{t=1}^{n}b_{it}a_{tj}&=\updelta_{i1}a_{1j}+\dotsb+\updelta_{ii}a_{ij}+\dotsb+\updelta_{in}a_{nj}=\\&=0+\dotsb+1\cdot a_{ij}+\dotsb+0=\\&=a_{ij}\end{aligned}\)
	\end{enumerate}
	
	Таким образом, \(c_{ij}=\begin{cases}\upalpha a_{ij},&i=k,\,j=\overline{1,n}\\a_{ij},&i\neq k,\,j=\overline{1,n}\end{cases}\)\end{proof}
	\ex
	
	\(\begin{aligned}&A=\begin{pmatrix}1&2&3\\4&5&6\\7&8&9\end{pmatrix}\oversymbolnoref[\upalpha = c]{k=2}{\sim}\begin{pmatrix}1&2&3\\4c&5c&6c\\7&8&9\end{pmatrix}\\
	&B=\begin{pmatrix}1&0&0\\0&c&0\\0&0&1\end{pmatrix}\implies B\cdot A = \begin{pmatrix}1&0&0\\0&c&0\\0&0&1\end{pmatrix}\cdot \begin{pmatrix}1&2&3\\4&5&6\\7&8&9\end{pmatrix}=\begin{pmatrix}1&2&3\\4c&5c&6c\\7&8&9\end{pmatrix}\end{aligned}\)
	
	\subsubsection{Перемена местами \(\boldsymbol{l}\)--ой и \(\boldsymbol{k}\)--ой строк}
	\lm{Соответствует левостороннему домножению на \(\boldsymbol{{B\left\{b_{ij}\right\}_{i,j=1}^{n}}}\), \(\boldsymbol{{b_{ij}=\begin{cases}\updelta_{ij},&i\neq k,\,i\neq l,\,j=\overline{1,n}\\\updelta_{kj},&i=l,\,j=\overline{1,n}\\\updelta_{lj},&i=k,\,j=\overline{1,n}\end{cases}}}\)}
	
	\begin{proof} \(B\cdot A = C\)
	
	\begin{enumerate}[{2.}1)]
		\item\(\displaystyle\big(i\neq k,\,i\neq l,\,j=\overline{1,n}\big)\text{ } c_{ij}=\sum_{t=1}^{n}b_{it}a_{tj}=0+\dotsb+0+\updelta_{ii}a_{ij}=a_{ij}\)
	
		\item\(\displaystyle\big(i=l,\,j=\overline{1,n}\big)\text{ }c_{lj}=\sum_{t=1}^{n}b_{lt}a_{tj}=0+\dotsb+0+\updelta_{kk}a_{kj}=a_{kj}\)
		
		\item\(\displaystyle\big(i=k,\,j=\overline{1,n}\big)\text{ }c_{kj}=\sum_{t=1}^{n}b_{kt}a_{tj}=0+\dotsb+0+\updelta_{ll}a_{lj}=a_{lj}\)
	\end{enumerate}

	Таким образом, \(c_{ij}=\begin{cases}a_{ij},&i\neq k,\,i\neq l,\,j=\overline{1,n}\\a_{kj},&i=l,\,j=\overline{1,n}\\a_{lj},&i=k,\,j=\overline{1,n}\end{cases}\)\end{proof}
	
	\ex

	\(\begin{aligned}&A=\begin{pmatrix}1&2&3\\4&5&6\\7&8&9\end{pmatrix}\oversymbolnoref[k=2]{l=1}{\sim}\begin{pmatrix}4&5&6\\1&2&3\\7&8&9\end{pmatrix}\\
	&B=\begin{pmatrix}0&1&0\\1&0&0\\0&0&1\end{pmatrix}\implies B\cdot A = \begin{pmatrix}0&1&0\\1&0&0\\0&0&1\end{pmatrix}\cdot\begin{pmatrix}1&2&3\\4&5&6\\7&8&9\end{pmatrix}=\begin{pmatrix}4&5&6\\1&2&3\\7&8&9\end{pmatrix}\end{aligned}\)
	
	\subsubsection{Добавление к \(\boldsymbol{k}\)--ой строке \(\boldsymbol{l}\)--ую, умноженную на \(\boldsymbol{\upalpha\in\mathbb{R}}\)}
	\lm{Соответствует левостороннему домножению на \(\boldsymbol{{B\left\{b_{ij}\right\}_{i,j=1}^{n}}}\), \(\boldsymbol{{b_{ij}=\begin{cases}\upalpha+\updelta_{kl},&i=k\text{ и } j=l\\\updelta_{ij},&i\neq k\text{ или }j\neq l\end{cases}}}\)}
	
	\begin{proof} \(B\cdot A = C\)
	
	\begin{enumerate}[{3.}1)]
		\item\(\displaystyle\big(i=k\neq l,\,j=\overline{1,n}\big)\text{ }c_{kj}=\sum_{t=1}^{n}b_{kt}a_{tj}=0+\dotsb+0+\updelta_{kk}a_{kj}+a_{lj}(\upalpha + 0)=a_{kj}+\upalpha a_{lj}\)
		
		\item\(\displaystyle\big(i=k=l=q,\,j=\overline{1,n}\big)\text{ } c_{kj}=c_{lj}=c_{qj}=\sum_{t=1}^{n}b_{qt}a_{tj}=0+\dotsb+0+a_{qj}(\upalpha+1)=a_{qj}+\upalpha a_{qj}\)
		
		\item\(\displaystyle\big(i\neq k,\,j=\overline{1,n}\big)\text{ } c_{ij}=\sum_{t=1}^{n}b_{it}a_{tj}=\sum_{t=1}^{n}b_{it}a_{tj}=0+\dotsb+0+\updelta_{ii}a_{ij}=a_{ij}\)
	\end{enumerate}

	Таким образом, \(c_{ij}=\begin{cases}a_{kj}+\upalpha a_{lj},&i=k,\,j=\overline{1,n}\\a_{ij},&i\neq k,\,j=\overline{1,n}\end{cases}\)\end{proof}
	\ex
	
	\(\begin{aligned} &A=\begin{pmatrix}1&2&3\\4&5&6\\7&8&9\end{pmatrix}\oversymbolnoref[k=2]{l=1}{\sim}\begin{pmatrix}1&2&3\\4+\upalpha&5+2\upalpha&6+3\upalpha\\7&8&9\end{pmatrix}\\
	&B=\begin{pmatrix}1&0&0\\\upalpha+\updelta_{21}&1&0\\0&0&1\end{pmatrix}\implies B\cdot A=\begin{pmatrix}1&0&0\\\upalpha&1&0\\0&0&1\end{pmatrix}\cdot\begin{pmatrix}1&2&3\\4&5&6\\7&8&9\end{pmatrix}=\begin{pmatrix}1&2&3\\4+\upalpha&5+2\upalpha&6+3\upalpha\\7&8&9\end{pmatrix}\end{aligned}\)
	
	\ex
	
	\(\begin{aligned}&A=\begin{pmatrix}1&2&3\\4&5&6\\7&8&9\end{pmatrix}\oversymbolnoref[k=3]{l=3}{\sim}\begin{pmatrix}1&2&3\\4&5&6\\7+7\upalpha&8+8\upalpha&9+9\upalpha\end{pmatrix}\\
	&B=\begin{pmatrix}1&0&0\\0&1&0\\0&0&\upalpha+\updelta_{33}\end{pmatrix}\implies B\cdot A=\begin{pmatrix}1&0&0\\0&1&0\\0&0&\upalpha+1\end{pmatrix}\cdot\begin{pmatrix}1&2&3\\4&5&6\\7&8&9\end{pmatrix}=\begin{pmatrix}1&2&3\\4&5&6\\7(\upalpha+1)&8(\upalpha+1)&9(\upalpha+1)\end{pmatrix}\end{aligned}\)
	
	Таким образом, последовательное применение ЭП строк к матрице \(A\) эквивалентно последовательному домножению слева матрицы \(A\) на соответствующую матрицу \(B_{i}\).
	
	\subsubsection{Метод элементарных преобразований строк}
	
	Пусть имеется некая квадратная матрица \(A\). Изобразим её и единичную матрицу: \[\begin{pmatrix}[c|c]A&E\end{pmatrix}\]
	
	Производя ЭП строк над обеими матрицами, добьёмся того, чтобы слева образовалась единичная матрица. Тогда справа образуется матрица, обратная \(A\): \[\begin{pmatrix}[c|c]A&E\end{pmatrix}\sim\begin{pmatrix}[c|c]E&A^{-1}\end{pmatrix}\]
	
	\begin{proof} По доказанному выше, применение ЭП строк эквивалентно левому домножению на некоторую матрицу \(B_{i}\). Таким образом, получим:
	
	\begin{enumerate}[{4.}1)]
		\item \(A\sim B_{m}\dotsb B_{n}A=E\implies \big(B_{m}\dotsb B_{n}\big) = A^{-1}\)
		\item \(E\sim B_{m}\dotsb B_{n}E = \big(B_{m}\dotsb B_{n}\big)\implies E\sim A^{-1}\)
	\end{enumerate}\end{proof}
	\ex
	
	\[\begin{split}&A=\begin{pmatrix}1&2\\1&3\end{pmatrix}\\ &\begin{pmatrix}[cc|cc]1&2&1&0\\1&3&0&1\end{pmatrix}\sim\begin{pmatrix}[cc|rr]1&2&1&0\\0&1&-1&1\end{pmatrix}\sim\begin{pmatrix}[cc|rr]1&0&3&-2\\0&1&-1&1\end{pmatrix}\end{split}\]
	
	Таким образом, \(A^{-1}=\begin{pmatrix}3&-2\\-1&1\end{pmatrix}\)
	
	\[A\cdot A^{-1}=\begin{pmatrix}1&2\\1&3\end{pmatrix}\cdot\begin{pmatrix}3&-2\\-1&1\end{pmatrix}=\begin{pmatrix}3-2&-2+2\\3-3&-2+3\end{pmatrix}=\begin{pmatrix}1&0\\0&1\end{pmatrix}=E\]
	
	\chapter{Ранг матрицы}
	\section{Определение ранга матрицы}
	\subsection{Минор}
	\df[Минор]
	
	В матрице \(A\) выделим строки \(i_1,\,i_2,\dotsc,i_r\) и столбцы \(j_1,\,j_2,\dotsc,j_r\). Определитель, составленный из выделенных элементов матрицы \(A\), называется минором \(\mathrm{M}_{_{\inlineperm{1}{r}}}^{^{\inlineperm[j]{1}{r}}}\).
	
	Порядок минора --- количество \(r\) выделенных строк и столбцов. 
	
	Для краткости некоторый минор порядка \(r\) соответствующей матрицы \(A\) будем обозначать как \(\mathrm{M}^{r}\big(A\big)\), а соответствующую матрицу минора как \(A\big(\mathrm{M}^{r}\big)\).
	
	\ex
	
	\(A=\begin{pmatrix}1&2&3&0\\4&5&6&0\\7&8&9&0\end{pmatrix}\)
	
	\(\mathrm{M}_{_{1\,2}}^{^{2\,3}}\big(A\big)=\begin{vmatrix}\tikzmarkemptyX{L11}{1}&\tikzmarkemptyX{C11}{2}&\tikzmarkemptyX{C21}{3}&\tikzmarkemptyX{L12}{0}\\\tikzmarkemptyX{L21}{4}&5&6&\tikzmarkemptyX{L22}{0}\\7&\tikzmarkemptyX{C12}{8}&\tikzmarkemptyX{C22}{9}&0\end{vmatrix}=\begin{vmatrix}2&3\\5&6\end{vmatrix}=12-15=-3
	\begin{tikzpicture}[overlay, remember picture]
		\draw[rounded corners, green, thick, fill=green!15, fill opacity=0.4] (L11.north west) rectangle (L12.south east);
		\draw[rounded corners, green, thick, fill=green!15, fill opacity=0.4] (L21.north west) rectangle (L22.south east);
		\draw[rounded corners, red, thick, fill=red!15, fill opacity=0.4] (C11.north west) rectangle (C12.south east);
		\draw[rounded corners, red, thick, fill=red!15, fill opacity=0.4] (C21.north west) rectangle (C22.south east);
	\end{tikzpicture}\)
	
	\(A\Big(\mathrm{M}_{_{1\,2}}^{^{2\,3}}\Big)=\begin{pmatrix}2&3\\5&6\end{pmatrix}\)
	
	Обратим внимание читателя на то, что понятие \textit{минора} матрицы \textbf{отличается} от понятия \hyperref[2.5.1]{\textit{\color{MMagenta}дополнительного минора}} матрицы, вводившегося ранее. 
	
	В частности, минор некого порядка можно вычислить для любой матрицы, а дополнительный минор --- только для квадратной.
	\subsection{Ранг матрицы}
	\df[Ранг матрицы]
	
	Ранг матрицы \(A\) --- максимальный порядок отличного от нуля минора.
	
	То есть, если \(\rank A=r\), то \(\exists\left\{i_1,\dotsc,i_r\right\}\text{ и }\left\{j_1,\dotsc,j_r\right\}: \mathrm{M}_{_{\inlineperm{1}{r}}}^{^{\inlineperm[j]{1}{r}}}\neq 0\), а любой другой минор более высокого порядка равен нулю.
	\section{Влияние элементарных преобразований строк на ранг матрицы}
	\tm[1]{При ЭП строк ранг матрицы не увеличивается}\label{4.2.1}
	\begin{proof} Пусть \(\rank A=r\).
	
	\begin{enumerate}[1)]
		\item Переставим строки \(l\) и \(k\). \(A\rightarrow B\).
		\begin{itemize}
			\item Рассмотрим минор матрицы \(B\) порядка \(r+1\). Он либо совпадает с соответствующим минором матрицы \(A\), либо отличается порядком строк. 
			
			Отсюда, \(\mathrm{M}^{r+1}\big(B\big)=\pm\mathrm{M}^{r+1}\big(A\big)=0\implies\rank B\leq r\).
		\end{itemize}
		\item Умножим \(k\)--ю строку \(A\) на \(\uplambda\in\mathbb{R}\). \(A\rightarrow B\).
		\begin{itemize}
			\item Рассмотрим минор матрицы \(B\) порядка \(r+1\). Он либо совпадает с минором \(A\) (если \(k\)--я строка не вошла), либо отличается в \(\uplambda\) раз.
			
			Отсюда, \(\mathrm{M}^{r+1}\big(B\big)=\left[\begin{aligned}&\uplambda\cdot\mathrm{M}^{r+1}\big(A\big)=0\\[2.5pt]&\mathrm{M}^{r+1}\big(A\big)=0\end{aligned}\right.\implies \rank B\leq r\).
		\end{itemize}
		\item Прибавим к \(k\)--ой строке \(l\)--ю, умноженную на \(\uplambda\in\mathbb{R}\). \(A\left\{a_{ij}\right\}\rightarrow B\left\{b_{ij}\mid b_{kj}=a_{kj}+\uplambda a_{lj}\right\}\).
		\begin{itemize}
			\item Рассмотрим минор матрицы \(B\) порядка \(r+1\):
			\begin{enumerate}[a)]
				\item Если \(k\)--я строка не вошла, то \(\mathrm{M}^{r+1}\big(B\big)=\mathrm{M}^{r+1}\big(A\big)=0\).
				\item Если \(k\)--я строка вошла, то из \hyperref[2.3]{\textit{\color{MMagenta}лемм о влиянии ЭП на определитель матрицы}} получим \(\mathrm{M}^{r+1}\big(B\big)=\left|B\big(\mathrm{M}^{r+1}\big)\right|=\left|A\big(\mathrm{M}^{r+1}\big)\right|+\uplambda\cdot 0=0\).
			\end{enumerate} Отсюда, \(\rank B\leq r\).
		\end{itemize}
	\end{enumerate}\end{proof}
	\tm[2]{При ЭП строк ранг матрицы не изменяется}
	
	\begin{proof} ЭП обратимы.
	\begin{itemize}
		\item \(A\xrightarrow{\text{ЭП}}B\). По \hyperref[4.2.1]{\color{MMagenta}\(\mathbf{\Big[\textbf{Th}^{1}\Big]}\)} \(\rank B\leq\rank A\)
		\item \(B\xrightarrow{\text{ЭП}}A\). По \hyperref[4.2.1]{\color{MMagenta}\(\mathbf{\Big[\textbf{Th}^{1}\Big]}\)} \(\rank A\leq\rank B\)
	\end{itemize}

	Отсюда, \(\rank B=\rank A\).\end{proof}
	
	\newpage\section{Вычисление ранга с помощью ЭП строк}
	\tm[3]{У матрицы ступенчатого вида ранг равен числу ненулевых строк}
	
	\begin{proof} Пусть у матрицы ступенчатого вида \(r\) ненулевых строк (без ограничения общности, пусть это строки \(1,\dotsc,r\)). Пусть тогда \(k_1,\dotsc,k_r\) --- ненулевые столбцы, в которых стоят начала ступенек.
	\[\mathrm{M}_{_{1\dotsb r}}^{^{\inlineperm[k]{1}{r}}}=\begin{vmatrix}a_{1k_1}&\hcdotsfor{4}\\
																		0&a_{2k_2}&\hcdotsfor{3}\\
																		0&0&\ddots&\hcdotsfor{2}\\
																		\vdots&\vdots&\vdots&\ddots&\vdots\\
																		0&\hcdotsfor{2}&0&a_{rk_r}\\													
	\end{vmatrix}=\prod_{n=1}^{r}a_{nk_n}\neq 0\]
	\\
	Это минор порядка \(r\). Любой другой минор большего порядка равен нулю.
	\end{proof}
	
	
	Таким образом, приводя матрицу ЭП строк к ступенчатому виду, можно определить её ранг.
	
	\section{Свойство базисных строк и базисных столбцов матрицы}
	\subsection{Базисный минор матрицы}
	\df[Базисный минор, базисные строки/столбцы]
	
	 Если для матрицы \(A\), \(\rank A=r\), то \(\mathrm{M}^{r}(A)\neq 0\) --- базисный минор.
	 
	 Строки и столбцы матрицы \(A\), входящие в базисный минор, называются базисными строками и базисными столбцами соответственно.
	 
	 \lm[1]{Для квадратной матрицы \(\bm{A}\), \(\bm{{|A|\neq 0}}\bm{{\iff}}\text{строки \(\bm{A}\) линейно независимы}\)}\label{4.4.1}\begin{proof}\(\langle\Rightarrow\rangle\) Предположим противное: существуют линейно зависимые строки. 
	 
	 Пусть всего \(n\) строк: \(\vv*{\mathbf{A}}{1},\dotsb,\vv*{\mathbf{A}}{m},\dotsb,\vv*{\mathbf{A}}{n}\). Пусть \(\displaystyle\vv*{\mathbf{A}}{m}=\sum_{\substack{k=1,\\[.5pt]k\neq m}}^{n}\upalpha_{k}\vect[l*]{A}{k}\). 
	 
	 Вычитая поочерёдно из этой строки все \(\upalpha_{k}\vect[l*]{A}{k}\left(k=\overline{1,n},\,k\neq m\right)\), произведём над этой строкой ЭП--я и в итоге получим нулевую строку \(\implies |A|=0\implies\skull\)
	 
	 Значит, все строки линейно независимы.\end{proof}
	 
	 \begin{proof} \(\langle\Leftarrow\rangle\) Приведём матрицу к ступенчатому виду.
	 
	 Нулевых строк нет (они линейно независимы), поэтому матрица имеет верхнетреугольный вид \(\displaystyle\implies a_{kk}\neq 0\left(\forall k=\overline{1,n}\right)\implies |A|=\prod_{k=1}^{n}a_{kk}\neq 0\)\end{proof}
	 
	 \newpage\tm[1]{Строки, входящие в базисный минор, линейно независимы}\label{4.4 Th1}
	 Пусть \(\rank A =r\implies\exists\mathrm{M}^{r}(A)\neq 0\). Пусть, для определённости, это минор \(\mathrm{M}_{_{1\dotsb r}}^{^{1\dotsb r}}\).
	 \begin{proof} По \hyperref[4.4.1]{\color{MMagenta}\(\mathbf{\Big[\textbf{Л}^1\Big]}\)}, т.к. \(\mathrm{M}_{_{1\dotsb r}}^{^{1\dotsb r}}\neq 0\), то строки матрицы минора будут линейно независимы.
	 
	 Теперь рассмотрим матрицу из первых \(r\) строк матрицы \(A\). Докажем, что они линейно независимы.
	 \[A=\begin{pmatrix}\tikzmarkempty{A}{a_{11}}&\cdots&a_{1r}&\cdots&a_{1n}\\
	 					\vdots&\ddots&\vdots&\ddots&\vdots\\
 						a_{r1}&\cdots&\tikzmarkempty{B}{a_{rr}}&\cdots&\tikzmarkempty{C}{a_{rn}}\\
 						\vdots&\ddots&\vdots&\ddots&\vdots\\
 						a_{m1}&\hcdotsfor{3}&a_{mn} \end{pmatrix}\begin{tikzpicture}[overlay, remember picture]
 						\draw[rounded corners=1.5pt, blue, fill=blue!5, fill opacity=0.4, very thin] (A.north west) rectangle (C.south east);
 						\draw[dashed, red, thin] (A.north west) rectangle (B.south east);
 						\end{tikzpicture}\]
	 
	 Предположим противное: одна из этих строк (пусть \(r\)--я) является линейной комбинацией других.
 	\[\forall k=\overline{1,n}:~a_{rk}=\sum_{i=1}^{r-1}\uplambda_{i}a_{ik}\text{, причём }r\leq n\]
 	
 	Это будет верно и для \(\forall k=\overline{1,r}\). Следовательно, \(r\)--я строка матрицы минора \(\mathrm{M}^{^{1\dotsb r}}_{_{1\dotsb r}}\) является линейной комбинацией других\(\implies\skull\).
 	
 	Значит, строки, входящие в базисный минор, линейно независимы.
 	\end{proof}
 
 	(То же самое относится и к базисным столбцам).
 	
 	\lm[2]{Если в матрице \(\bm{A}\) есть \(\bm{m}\) штук линейно независимых строк, то существует\nolinebreak[4] \(\bm{\mathrm{M}^{m}\big(A\big)\neq 0}\)}\label{4.4 L2}
 	\begin{proof}Приведём эту матрицу к ступенчатому виду. Найдутся \(l_{1},\dotsc,l_{m}\) и \(k_{1},\dotsc,k_{m}\) -- ненулевые строки и столбцы соответственно. Таким образом, ранг матрицы \(A\) равен \(m\) и нужный минор имеет вид \(\mathrm{M}_{_{l_{1}\dotsb l_{m}}}^{^{k_{1}\dotsb k_{m}}}=\mathrm{M}^{m}\big(A\big)\neq 0\).
 	\end{proof}
 
  	Далее будем пользоваться следующими обозначениями:
 	\begin{align*}
 	\vect[l]{A}{}=\begin{pmatrix}a_{1}&\cdots&a_{n}\end{pmatrix}&\text{ -- вектор-строка}
 	\\	
 	\vect[c]{X}{}=\begin{pmatrix}x_{1}\\\vdots\\x_{n}\end{pmatrix}&\text{ -- вектор-столбец}
 	\\
 	\vect{b}{}&\text{ -- любой вектор (вектор-строка ИЛИ вектор-столбец)}
 	\end{align*}
 
 	\df[Линейная оболочка системы векторов]
 	
 	Линейной оболочкой данной системы векторов \(\{\vect{a}{1},\dotsc,\vect{a}{n}\}\) называется множество
 	
 	\[L\big(\vect{a}{1}\dotsb\vect{a}{n}\big)=\Bigg\{\vect{x}{}~\bigg|~\exists\{\uplambda_{i}\}_{i=1}^{n}\subset\mathbb{R}: \vect{x}{}=\sum_{i=1}^{n}\uplambda_{i}\vect{a}{i}\Bigg\}\] 
 	
 	\lm[3]{Если система векторов \(\boldsymbol{\{}\vect{a}{\bm{1}}\boldsymbol{,}\boldsymbol{\dotsc}\,\boldsymbol{,}\vect{a}{\bm{k}}\boldsymbol{\}}\) -- линейно независима, а \(\boldsymbol{\{}\vect{a}{\bm{1}}\boldsymbol{,\dotsc\,,}\vect{a}{\bm{k}}\boldsymbol{\}}\boldsymbol{\cup}\boldsymbol{\{}\vect{b}{}\boldsymbol{\}}\)\nolinebreak[4] --\nolinebreak[4] линейно зависима, то \(\vect{b}{}\boldsymbol{\in} \boldsymbol{L\big(}\vect{a}{\bm{1}}\boldsymbol{\dotsb}\,\vect{a}{\bm{k}}\bm{\big)}\)}\label{4.4 L3}
 	\begin{proof}\(\{\vect{a}{1},\dotsc,\vect{a}{k},\vect{b}{}\}\) -- линейно зависимы\(\displaystyle\iff\exists\{\uplambda_i\}_{i=1}^{k},\upmu:\begin{aligned}[t]&\sum_{i=1}^{k}|\uplambda_i|+|\upmu|\neq 0\text{ и }\\ &\sum_{i=1}^{k}\uplambda_{i}\vect{a}{i}+\upmu\vect{b}{}=\vect{0}{}\end{aligned}\)
 	\begin{enumerate}[1)]\item Если \(\upmu=0\), то \(\{\vect{a}{1},\dotsc,\vect{a}{k}\}\) -- линейно зависимы, что противоречит условию. \item Если \(\upmu\neq 0\), то, поделив сумму на \(\upmu\neq 0\), получим \(\vect{b}{}\in L\big(\vect{a}{1}\dotsb\vect{a}{k}\big)\).\end{enumerate}\end{proof}
 	
 	\tm[2]{Если \(\vect[l*]{A}{\boldsymbol{m_1}}\boldsymbol{,\dotsc\,,}\vect[l*]{A}{\boldsymbol{m_r}}\) --- строки, входящие в базисный минор матрицы \(\boldsymbol{A}\) (т.е.\nolinebreak[4] \(\boldsymbol{\rank A = r}\)), то \(\boldsymbol{\big(\forall k\neq m_{i},\, i=\overline{1,r}\big)}\vect[l*]{A}{\boldsymbol{k}}\boldsymbol{\in} \boldsymbol{L\Big(}\vect[l*]{A}{\boldsymbol{m_1}}\boldsymbol{\dotsb}\,\vect[l*]{A}{\boldsymbol{m_r}}\boldsymbol{\Big)}\)}
 	\begin{proof}Докажем, что \(\Big\{\vect[l*]{A}{m_1},\dotsc,\vect[l*]{A}{m_r},\vect[l*]{A}{k}\Big\}\) --- линейно зависимая система. 
 		
 	Предположим противное: тогда имеем в матрице \(A\) \((r+1)\) штук линейно независимых строк. По \hyperref[4.4 L2]{\color{MMagenta}\(\mathbf{\Big[\textbf{Л}^2\Big]}\)} можно найти минор \(\mathrm{M}^{r+1}\big(A\big)\neq 0\), т.е. \(\rank A\neq r\implies\skull\).
 
	Следовательно, \(\Big\{\vect[l*]{A}{m_1},\dotsc,\vect[l*]{A}{m_r}\Big\}\bigcup\Big\{\vect[l*]{A}{k}\Big\}\) --- лин. зав., но из \hyperref[4.4 Th1]{\color{MMagenta}\(\mathbf{\Big[\textbf{Th}^1\Big]}\)} \(\Big\{\vect[l*]{A}{m_1},\dotsc,\vect[l*]{A}{m_r}\Big\}\)~---~лин.~нез. Значит, по \hyperref[4.4 L3]{\color{MMagenta}\(\mathbf{\Big[\textbf{Л}^3\Big]}\)} \(\vect[l*]{A}{k}\in L\Big(\vect[l*]{A}{m_1}\dotsb\,\vect[l*]{A}{m_r}\Big)\).\end{proof}

	\chapter{Системы линейных алгебраических уравнений}
	\[\left\{\begin{array}{@{}l@{}}a_{11}x_{1}+a_{12}x_{2}+\dotsb+a_{1n}x_{n}=b_{1},\\[3pt]
								  \hcdotsfor{1}\\[3pt]
							  	  a_{m1}x_{1}+a_{m2}x_{2}+\dotsb+a_{mn}x_{n}=b_{m}.\end{array}\right.\]
	\vspace*{-1.25\baselineskip}\begin{center}\footnotesize Система линейных алгебраических уравнений\\
					на неизвестные \(\{x_{1},x_{2},\dotsc,x_{n}\}=\vect{x}{}\)\end{center}
	\section{Решение, совместность, определённость}
	\df[Решение СЛАУ]
	
	Решением системы линейных алгебраических уравнений называется совокупность \(n\) штук значений неизвестных \(x_1=\upalpha_1,\, x_2=\upalpha_2,\,\dotsc,\, x_n=\upalpha_n\), при подстановке которых все уравнения системы обращаются в тождества.
	\df[Совместная и несовместная СЛАУ]
	
	Совместной называется система, имеющая хотя бы одно решение; не имеющая --- несовместной.
	\df[Определённость и неопределённость СЛАУ]
	
	Определённой называется система, имеющая единственное решение; имеющая более одного \nolinebreak--- неопределённой.
	\newpage\section[Влияние ЭП на решение СЛАУ]{Влияние элементарных преобразований на решение системы линейных алгебраических уравнений}
	
	Переформулируем уже знакомые элементарные преобразования с языка матриц на язык СЛАУ:
	\begin{enumerate}[1)]\item Перестановка уравнений; \item Умножение уравнения на \(\uplambda\neq 0\); \item Прибавление к \(k\)--му уравнению \(l\)--го умноженного на \(\upmu\).\end{enumerate}
	
	\lm{Если ЭП--ем перейти от системы \(\bm{A}\) к системе \(\bm{B}\), то любое решение \(\bm{A}\) будет являться решением \(\bm{B}\)}\label{5.2 L}
	\begin{proof}
	Пусть \(x_1=x_{1}^{(0)},\, x_2=x_2^{(0)},\dotsc,\, x_n=x_n^{(0)}\) --- некоторое решение системы \(A\).
	
	\begin{enumerate}[(1)]\item Справедливость леммы очевидна;
						  \item Пусть изменили \(k\)--ое уравнение:
								\begin{tabbing}В системе \(A\): \=\(\displaystyle\sum_{t=1}^{n}a_{kt}x_{t}^{(0)}=b_k\).\\[5pt] В системе \(B\): \>\(\displaystyle\sum_{t=1}^{n}(\uplambda a_{kt})x_{t}=\uplambda b_k\). Подставим решение \(A\):\\[5pt] \>\(\displaystyle\uplambda\cdot\sum_{t=1}^{n}a_{kt}x_{t}^{(0)}\mathbin{\oversymbolnoref{\;A\;}{=}}\uplambda\cdot b_k\) --- решение \(B\);\end{tabbing}
						  \item Пусть изменили \(k\)--ое уравнение:
						  		\begin{tabbing}В системе \(A\): \=\(\displaystyle\sum_{t=1}^{n}a_{kt}x_{t}^{(0)}=b_k\text{ и }\sum_{t=1}^{n}a_{lt}x_{t}^{(0)}=b_l\).\\[5pt]В системе \(B\): \>\(\displaystyle\sum_{t=1}^{n}\big(a_{kt}x_{t}+(\upmu a_{lt})x_{t}\big)=b_{k}+\upmu b_{l}\). Подставим решение \(A\):\\[5pt]\>\(\displaystyle\sum_{t=1}^{n}a_{kt}x_{t}^{(0)}+\upmu\sum_{t=1}^{n}a_{lt}x_{t}^{(0)}\mathbin{\oversymbolnoref{\;A\;}{=}}b_{k}+\upmu b_{l}\) --- решение \(B\).\end{tabbing}
	\end{enumerate}
	\end{proof}
	
	\tm{Если система \(\bm{B}\) получена из системы \(\bm{A}\) путём ЭП--ий, то они эквивалентны}\label{5.2 Th}
	\begin{proof} Поскольку ЭП--я обратимы, то утверждение следует из \hyperref[5.2 L]{\color{MMagenta}\(\mathbf{\Big[\textbf{Л}\Big]}\)}.\end{proof}
	
	\section[Теорема Крамера решения СЛАУ]{Теорема Крамера решения системы линейных алгебраических уравнений}
	\tm{Если в системе одинаковое количество уравнений и неизвестных, а определитель \(\bm{|A|}\) основной матрицы системы не равен нулю, то система является определённой}\label{5.3 Th}
	\[\left\{\begin{array}{@{}l@{}}
									a_{11}x_{1}+\dotsb+a_{1n}x_{n}=b_{1},\\[3pt]
									a_{21}x_{1}+\dotsb+a_{2n}x_{n}=b_{2},\\[3pt]
									\hcdotsfor{1}\\[3pt]
									a_{n1}x_{1}+\dotsb+a_{nn}x_{n}=b_{n}.
			 \end{array}\right.\]
		 
	\[A=\begin{pmatrix}a_{11}&\cdots&a_{1n}\\
					   a_{21}&\cdots&a_{2n}\\
				   	   \hcdotsfor{3}\\
			   	   	   a_{n1}&\cdots&a_{nn}\end{pmatrix},\, \vect[c]{X}{}=\begin{pmatrix}x_{1}\\\vdots\\x_{n}\end{pmatrix},\,\vect[c]{B}{}=\begin{pmatrix}b_1\\\vdots\\b_n\end{pmatrix}\]
	\begin{proof}Введём \(\Updelta_{i}\):
	\begin{align*} \Updelta_{1}&=\begin{vmatrix}\vect[c]{B}{}&\vect[c]{A}{2}&\cdots&\vect[c]{A}{n}\end{vmatrix}=\begin{vmatrix}b_{1}&a_{12}&\tikzmarkempty{A}{\phantom{a_{22}}}&a_{1n}\\ \vdots&\vdots&\phantom{\vdots}&\vdots\\b_{n}&a_{n2}&\tikzmarkempty{B}{\phantom{a_{22}}}&a_{nn}\end{vmatrix},\\[10pt]
	\cdots\\[10pt]
	\Updelta_{i}&=\begin{vmatrix}\vect[c]{A}{1}&\cdots&\vect[c]{A}{i-1}&\vect[c]{B}{}&\vect[c]{A}{i+1}&\cdots&\vect[c]{A}{n}\end{vmatrix}=\begin{vmatrix}a_{11}&\tikzmarkempty{D}{\phantom{a_{12}}}&a_{1(i-1)}&b_{1}&a_{1(i+1)}&\tikzmarkempty{C}{\phantom{a_{12}}}&a_{1n}\\
				\vdots&\phantom{\vdots}&\vdots&\vdots&\vdots&\phantom{\vdots}&\vdots\\
				a_{n1}&\tikzmarkempty{De}{\phantom{a_{n2}}}&a_{n(i-1)}&b_n&a_{n(i+1)}&\tikzmarkempty{Ce}{\phantom{a_{nn}}}&a_{nn}\end{vmatrix}\text{ и т.д.}
	\end{align*}
	\begin{tikzpicture}[overlay, remember picture]
		\draw[line width=1.157pt, line cap=round, dash pattern=on 0pt off 3.5\pgflinewidth] (A.north) -- (B.south);
		\draw[line width=1.157pt, line cap=round, dash pattern=on 0pt off 3.5\pgflinewidth] (D.north) -- (De.south);
		\draw[line width=1.157pt, line cap=round, dash pattern=on 0pt off 3.5\pgflinewidth] (C.north) -- (Ce.south);
	\end{tikzpicture}
	\begin{enumerate}[1)]\item\(A\cdot\vect[c]{X}{}=\vect[c]{B}{}\)
		
		\begin{minipage}[t]{.5\textwidth}\(\vect[c]{X}{}=A^{-1}\cdot\vect[c]{B}{}\), пусть \(A^{-1}=C\{c_{ij}\}_{i,j=1}^{n}\), \(c_{ij}=\dfrac{\mathrm{A}_{ji}}{\Updelta}\), \end{minipage}%
		\begin{minipage}[t]{.5\textwidth}\begin{tabbing}где \=\(\mathrm{A}_{ji}\) --- \(\begin{aligned}[t]&\text{алгебраическое дополнение}\\ &\text{соответствующего элемента}\\ &\text{матрицы }A^\mathrm{T},\end{aligned}\)\\[5pt]\>\(\Updelta=\det A\neq 0\).\end{tabbing}\end{minipage}
	
	Отсюда, \(\displaystyle\big(\forall p=\overline{1,n}\big)\, x_{p}=\sum_{t=1}^{n}c_{pt}b_{t}=\dfrac{1}{\Updelta}\cdot\underbrace{\sum_{t=1}^{n}\mathrm{A}_{tp}b_{t}}_{\mathclap{\substack{\text{Разложение }\Updelta_{p}\\\text{ по }p\text{--му столбцу}}}}\iff x_{p}=\dfrac{\Updelta_{p}}{\Updelta}\iff x_{p}\cdot\Updelta=\Updelta_{p}\)
	
	Убедимся, что полученная система эквивалентна исходной:
	
	\(\displaystyle x_{p}\cdot\bigg(\sum_{t=1}^{n}\mathrm{A}_{tp}a_{tp}\bigg)=\sum_{t=1}^{n}\mathrm{A}_{tp}b_{t}\iff x_{p}\cdot\bigg(\sum_{t=1}^{n}\mathrm{A}_{tp}a_{tp}\bigg)+\sum_{\substack{\phantom{,}s=1,\\s\neq p}}^{n}x_{s}\cdot\underbrace{\bigg(\sum_{t=1}^{n}\mathrm{A}_{tp}a_{ts}\bigg)}_{\mathclap{\substack{\hyperref[2.7.2 Th]{\color{MMagenta}\text{Фальшивое}}\text{ разложение }\Updelta\\\text{по }s\text{--му \emph{столбцу} --- равно нулю}}}}=\sum_{t=1}^{n}\mathrm{A}_{tp}b_{t}\iff\\[5pt]\iff\sum_{s=1}^{n}x_{s}\cdot\bigg(\sum_{t=1}^{n}\mathrm{A}_{tp}a_{ts}\bigg)=\sum_{t=1}^{n}\mathrm{A}_{tp}b_{t}\iff \sum_{t=1}^{n}\mathrm{A}_{tp}\cdot\bigg(\sum_{s=1}^{n}x_{s}a_{ts}\bigg)=\sum_{t=1}^{n}\mathrm{A}_{tp}b_{t}\iff\\[8pt] \iff 
	\begin{aligned} 
		&\mathrm{A}_{1p}\cdot(x_{1}a_{11}+\dotsb+x_{n}a_{1n})+{}\\
		{}+{}&\mathrm{A}_{2p}\cdot(x_{1}a_{21}+\dotsb+x_{n}a_{2n})+{}\\
		{}+{}&\tikzmarkempty{As1}{}\phantom{\mathrm{A}_{3p}\cdot (x_{1}a_{31}+\dotsb+x_{n}a_{3n})}\tikzmarkempty{Bs1}{}{}+{}\\
		{}+{}&\mathrm{A}_{pp}\cdot(x_{1}a_{p1}+\dotsb+x_{n}a_{pn})+{}\\
		{}+{}&\tikzmarkempty{As2}{}\phantom{\mathrm{A}_{3p}\cdot (x_{1}a_{31}+\dotsb+x_{n}a_{3n})}\tikzmarkempty{Bs2}{}{}+{}\\
		{}+{}&\mathrm{A}_{np}\cdot(x_{1}a_{n1}+\dotsb+x_{n}a_{nn})
		\begin{tikzpicture}[overlay, remember picture]
		\draw[line width=1.157pt, line cap=round, dash pattern=on 0pt off 3.5\pgflinewidth] (As1.center) -- (Bs1.center);
		\draw[line width=1.157pt, line cap=round, dash pattern=on 0pt off 3.5\pgflinewidth] (As2.north) -- (Bs2.north);
		\end{tikzpicture}\end{aligned}=\begin{aligned}&\mathrm{A}_{1p}b_{1}+{}\\
												 {}+{}&\mathrm{A}_{2p}b_{2}+{}\\
											 	 {}+{}&\tikzmarkempty{Ai1}{}\phantom{\mathrm{A}_{3p}b_{3}}\tikzmarkempty{Bi1}{}+{}\\
											 	 {}+{}&\mathrm{A}_{pp}b_{p}+{}\\
											 	 {}+{}&\tikzmarkempty{Ai2}{}\phantom{\mathrm{A}_{3p}b_{3}}\tikzmarkempty{Bi2}{}+{}\\
											 	 {}+{}&\mathrm{A}_{np}b_{n}
										 	 	\begin{tikzpicture}[overlay, remember picture]
										 	 		\draw[line width=1.157pt, line cap=round, dash pattern=on 0pt off 3.5\pgflinewidth] (Ai1.north) -- (Bi1.north);
										 	 		\draw[line width=1.157pt, line cap=round, dash pattern=on 0pt off 3.5\pgflinewidth] (Ai2.north) -- (Bi2.north);
									 	 	\end{tikzpicture}\end{aligned}\)
	
	То есть преобразование выше можно рассматривать как ЭП--ия \(p\)--ой строки, а именно: умножение её на \(\mathrm{A}_{pp}\), затем прибавление к ней всех остальных строк, умноженных на \(\mathrm{A}_{tp}\) соответственно (\(t=\overline{1,n},\,t\neq p\)).
	
	Таким образом, получили новую систему  \(\vect[c]{X}{}=\left(\begin{smallmatrix}\sfrac{\Updelta_{1}}{\Updelta}\\ \sfrac{\Updelta_{2}}{\Updelta}\\[1pt]\myvdots\\[1pt]\sfrac{\Updelta_{n}}{\Updelta}\end{smallmatrix}\right)\) путём ЭП--ий исходной системы. По \hyperref[5.2 Th]{\color{MMagenta}\(\mathbf{\Big[\textbf{Th}\Big]}\)} они эквивалентны. Единственность решения доказана. \item Покажем, что полученное выражение --- решение исходной системы. Подставим выражение в любое уравнение системы (пусть \(q\)--ое, \(1\leq q\leq n\)):
	
	\(\displaystyle a_{q1}x_{1}+a_{q2}x_{2}+\dotsb+a_{qn}x_{n}=\sum_{p=1}^{n}a_{qp}x_{p}=\dfrac{1}{\Updelta}\sum_{p=1}^{n}a_{qp}\sum_{t=1}^{n}\mathrm{A}_{tp}b_{t}=\dfrac{1}{\Updelta}\sum_{t=1}^{n}b_{t}\underbrace{\sum_{p=1}^{n}a_{qp}\mathrm{A}_{tp}}_{\mathclap{\substack{\text{При }t\neq q\text{ это}\\\text{фальшивое разложение }\det A}}}=\\[5pt]=\underbrace{0}_{\mathclap{t=1\neq q}}+\underbrace{0}_{\mathclap{t=2\neq q}}+\dotsb+\underbrace{\dfrac{1}{\cancel{\Updelta}}\cdot b_{q}\cdot\cancel{\Updelta}}_{\mathclap{t=q}}+\underbrace{0}_{\mathclap{t=q+1}}+\dotsb+\underbrace{0}_{\mathclap{t=n\neq q}}=b_{q}\).\end{enumerate}
	\end{proof}
	\section[Структура общего решения однородной СЛАУ]{Структура общего решения однородной системы линейных алгебраических уравнений}
	\begin{equation} \left\{\begin{array}{@{}l@{}}a_{11}x_{1}+a_{12}x_{2}+\dotsb+a_{1n}x_{n}=0,\\[3pt]
		\hcdotsfor{1}\\[3pt]
		a_{m1}x_{1}+a_{m2}x_{2}+\dotsb+a_{mn}x_{n}=0.\end{array}\right.\label{HSOQm}\tag{1}\end{equation}
	\vspace*{-0.75\baselineskip}\begin{center}\footnotesize Однородная СЛАУ
		на неизвестные \(\{x_{1},x_{2},\dotsc,x_{n}\}=\vect{x}{}\)\end{center}
	
	Пусть ранг матрицы \(A=\begin{pmatrix}a_{11}&\cdots&a_{1n}\\
								\hcdotsfor{3}\\
								a_{m1}&\cdots&a_{mn}\end{pmatrix}\) равен \(r\). 
	Для определённости положим минор \nolinebreak[4]\(\mathrm{M}_{_{1\dotsb r}}^{^{1\dotsb r}}\neq 0\).
	
	Тогда, путём ЭП--ий строк данную систему можно привести к системе \begin{equation} \left\{\begin{array}{@{}l@{}}a_{11}x_{1}+a_{12}x_{2}+\dotsb+a_{1n}x_{n}=0,\\[3pt]
			\hcdotsfor{1}\\[3pt]
			a_{r1}x_{1}+a_{r2}x_{2}+\dotsb+a_{rn}x_{n}=0.\end{array}\right.\label{HSOQ}\tag{2}\end{equation}
	Причём, по \hyperref[5.2 Th]{\color{MMagenta}\(\mathbf{\Big[\textbf{Th}\Big]}\)} системы \eqref{HSOQm}---\eqref{HSOQ} эквивалентны.
	
	Представим систему \eqref{HSOQ} в виде \begin{equation}\left\{\begin{array}{@{}l@{}}a_{11}x_{1}+\dotsb+a_{1r}x_{r}=-a_{1(r+1)}x_{r+1}-\dotsb-a_{1n}x_{n},\\[3pt]
				\hcdotsfor{1}\\[3pt]
				a_{r1}x_{1}+\dotsb+a_{rr}x_{r}=-a_{r(r+1)}x_{r+1}-\dotsb-a_{rn}x_{n}.\end{array}\right.\label{HSOQ*}\tag{2*}\end{equation}
	
	Назовём \(\{x_{1},\dotsc,x_{r}\}\) \emph{базовыми} неизвестными, а \(\{x_{r+1},\dotsc,x_{n}\}\) --- \emph{свободными} неизвестными. Если задать свободные неизвестные, то система \eqref{HSOQ*} будет удовлетворять \hyperref[5.3 Th]{\color{MMagenta}теореме Крамера} и, следовательно, базовые неизвестные выразятся однозначно:
	\begin{align*} &\left\{\begin{array}{@{}l@{}} x_{r+1}=1,\\[5pt] x_{r+2}=x_{r+3}=\dotsb=x_{n}=0.\end{array}\right.\implies \vect[c]{X}{}^{(1)}=\begin{pmatrix}x_{1}^{(1)}\\[-1pt] \vdots\\[2pt]x_{r}^{(1)}\\1\\0\\[-3pt]\vdots\\0 \end{pmatrix} \\[5pt] &\cdots \\[5pt] &\left\{\begin{array}{@{}l@{}} x_{n}=1,\\[5pt] x_{r+1}=x_{r+2}=\dotsb=x_{n-1}=0.\end{array}\right.\implies\vect[c]{X}{}^{(n-r)}=\begin{pmatrix}x_{1}^{(n-r)}\\[-1pt] \vdots\\[2pt]x_{r}^{(n-r)}\\0\\0\\[-3pt] \vdots\\1 \end{pmatrix} \end{align*}
	
	Докажем ряд утверждений о системе вектор-столбцов \(\Big\{\vect[c]{X}{}^{(1)},\dotsc,\vect[c]{X}{}^{(n-r)}\Big\}\):
	\begin{flushleft}\encircle{\textbf{1}} \(\boldsymbol{\Big\{}\vect[c]{X}{}^{\boldsymbol{(1)}}\boldsymbol{,\dotsc\,,}\vect[c]{X}{}^{\boldsymbol{(n-r)}}\boldsymbol{\Big\}}\)\textbf{ --- линейно независимая система}\end{flushleft}
	
	\begin{proof}Рассмотрим матрицу \(U\), составленную из столбцов \(\vect[c]{X}{}^{(i)},\, i=\overline{1,n-r}\):\[U=\begin{pmatrix}x_{1}^{(1)}&x_{1}^{(2)}&\tikzmarkempty{Aa}{\phantom{x_{1}^{(3)}}}&x_{1}^{(n-r)}\\
					   \vdots     &\vdots     &\phantom{\vdots}                        &\vdots\\
				   	   x_{r}^{(1)}&x_{r}^{(2)}&\phantom{\vdots}						   &x_{r}^{(n-r)}\\
			   	   	   1		  &0		  &\phantom{\vdots}						   &0\\
		   	   	   	   0		  &1		  &\phantom{\vdots}						   &0\\
	   	   	   	   	   \vdots     &\vdots     &\phantom{\vdots}                        &\vdots\\
   	   	   	   	   	   0		  &0		  &\tikzmarkempty{Bb}{\phantom{1}}	   &1\end{pmatrix}
      	   	   	   	   	\begin{tikzpicture}[overlay, remember picture]
      	   	   	   	   	\draw[line width=1.157pt, line cap=round, dash pattern=on 0pt off 3.5\pgflinewidth] (Aa.north) -- (Bb.south);\end{tikzpicture}
         	   	   	   	\]
    \(\mathrm{M}^{^{1\dotsb n-r}}_{_{r+1\dotsb n}}\big(U\big)=1\neq 0\) --- минор порядка \((n-r)\).
    
    Поскольку минора большего порядка нет, то \(\rank U=n-r\implies \mathrm{M}^{^{1\dotsb n-r}}_{_{r+1\dotsb n}}\big(U\big)\) --- базисный минор. По \hyperref[4.4 Th1]{\color{MMagenta} свойству базисных столбцов} система \(\Big\{\vect[c]{X}{}^{(1)},\dotsc,\vect[c]{X}{}^{(n-r)}\Big\}\) линейно независима.
    \end{proof}

	\begin{flushleft}\encircle{\textbf{2}} \(\displaystyle \vect[c]{X}{}=\boldsymbol{\sum_{i=1}^{n-r}}\boldsymbol{\uplambda_{i}}\vect[c]{X}{}^{\boldsymbol{(i)}}\)\textbf{ --- решение системы \hyperref[HSOQ]{\(\boldsymbol{(2)}\)} при любом наборе \(\boldsymbol{\{\uplambda_j\}_{j=1}^{n-r}}\)}\end{flushleft}\label{5.4 S2}
	
	\begin{proof} \(\vect[c]{X}{}=\begin{pmatrix}x_{1}\\\vdots\\ x_{r}\\x_{r+1}\\x_{r+2}\\\vdots\\x_{n}\end{pmatrix}=\displaystyle \sum_{i=1}^{n-r}\uplambda_{i}\vect[c]{X}{}^{(i)}=\begin{pmatrix}\displaystyle \sum_{i=1}^{n-r}\uplambda_{i} x_{1}^{(i)}\\ \vdots\\\displaystyle \sum_{i=1}^{n-r}\uplambda_{i} x_{r}^{(i)}\\ \uplambda_{1}\\ \uplambda_{2}\\\vdots\\\uplambda_{n-r}\end{pmatrix}\implies\left\{\begin{array}{@{}l@{}}x_{1}=\displaystyle\sum_{i=1}^{n-r}\uplambda_{i} x_{1}^{(i)},\\ \hcdotsfor{1}\\ \displaystyle x_{r}=\sum_{i=1}^{n-r}\uplambda_{i}x_{r}^{(i)},\\ x_{r+1}=\uplambda_{1},\\ x_{r+2}=\uplambda_{2},\\ \hcdotsfor{1}\\ x_{n}=\uplambda_{n-r}. \end{array}\right.\)
	
	Подставим значения неизвестных в, например, \(k\)--ое (\(1\leq k\leq r\)) уравнение системы \eqref{HSOQ*}, которая эквивалентна системе \eqref{HSOQ}.
	\begin{align*}\sum_{j=1}^{r}a_{kj}x_{j}&\mathbin{\oversymbolnoref{\:\boldsymbol{?}\:}{=}}-\sum_{j=r+1}^{n}a_{kj}x_{j}\\[5pt] \sum_{j=1}^{r}a_{kj}\sum_{i=1}^{n-r}\uplambda_{i}x_{j}^{(i)}&\mathbin{\oversymbolnoref{\:\boldsymbol{?}\:}{=}}-\sum_{t=1}^{n-r}a_{k(r+t)}\uplambda_{t}\\[5pt] \sum_{i=1}^{n-r}\uplambda_{i}\sum_{j=1}^{r}a_{kj}x_{j}^{(i)}&\mathbin{\oversymbolnoref{\:\boldsymbol{?}\:}{=}}-\sum_{t=1}^{n-r}a_{k(r+t)}\uplambda_{t}\end{align*}
	По построению \(\big\{x_{j}^{(i)}\big\}_{j=1}^{r}\) получим
	\begin{align*}\sum_{i=1}^{n-r}\uplambda_{i}\cdot\big({-a_{k(r+i)}}\big)&\mathbin{\oversymbolnoref{\,\boldsymbol{\checkmark}\,}{=}}-\sum_{t=1}^{n-r}a_{k(r+t)}\uplambda_{t}\end{align*}\end{proof}
	
	\begin{flushleft}\encircle{\textbf{3}} \textbf{Любое решение системы \hyperref[HSOQm]{\(\boldsymbol{(1)}\)} можно получить как линейную комбинацию системы } \(\boldsymbol{\Big\{}\vect[c]{X}{}^{\boldsymbol{(1)}}\boldsymbol{,\dotsc\,,}\vect[c]{X}{}^{\boldsymbol{(n-r)}}\boldsymbol{\Big\}}\)\end{flushleft}
	
	\begin{proof}
	
	Пусть \(\vect[c]{Y}{}=\begin{pmatrix}y_{1}\\ \vdots\\ y_{n}\end{pmatrix}\) --- какое-нибудь решение системы \eqref{HSOQm}.
	
	Рассмотрим \(\displaystyle \vect[c]{Z}{}=\sum_{i=1}^{n-r}y_{r+i}\vect[c]{X}{}=\left(\begin{smallmatrix}z_{1}\\[1pt] \myvdots\\[1pt] z_{r}\\[2pt]y_{r+1}\\[1pt] \myvdots\\[1pt] y_{n}\end{smallmatrix}\right)\) --- решение системы \eqref{HSOQ} согласно утверждению \hyperref[5.4 S2]{\color{MMagenta}\encircle{\textbf{2}}}, а,\nolinebreak[4] соответственно, и системы \eqref{HSOQm}.

	Поскольку \(\vect[c]{Y}{}\) и \(\vect[c]{Z}{}\) оба являются решениями системы \eqref{HSOQm}, и \(\vect[c]{Z}{}\) совпадает с \(\vect[c]{Y}{}\) в разрядах от \((r+1)\)--го до \(n\)--го, то, подставив их в систему \eqref{HSOQ*}, можем приравнять левые части полученных выражений. Получим систему Крамера (СЛАУ, удовлетворяющую теореме Крамера) относительно \(\{z_{1},\dotsc,z_{r}\}\).
	\[\left\{\begin{array}{@{}l@{}}a_{11}z_{1}+\dotsb+a_{1r}z_{r}=a_{11}y_{1}+\dotsb+a_{1r}y_{r},\\ \hcdotsfor{1}\\ a_{r1}z_{1}+\dotsb+a_{rr}z_{r}=a_{r1}y_{1}+\dotsb+a_{rr}y_{r}.\end{array}\right.\]
	Она имеет единственное решение, которое выглядит как \(\left\{\begin{array}{@{}l@{}}z_{1}=y_{1},\\z_{2}=y_{2},\\ \hcdotsfor{1}\\ z_{r}=y_{r}.\end{array}\right.\) Отсюда, \(\vect[c]{Y}{}=\vect[c]{Z}{}\).

	Следовательно, \(\vect[c]{Y}{}\in L\Big(\vect[c]{X}{}^{(1)},\dotsc,\vect[c]{X}{}^{(n-r)}\Big)\) и система \(\Big\{\vect[c]{X}{}^{(1)},\dotsc,\vect[c]{X}{}^{(n-r)}\Big\}\) называется \emph{фундаментальной системой решений} (ФСР) СЛАУ \eqref{HSOQm}. \end{proof}

	\section[Структура общего решения неоднородной СЛАУ]{Структура общего решения неоднородной системы линейных алгебраических уравнений}
	\subsection{Теорема Кронекера--Капелли}
	\tm{Неоднородная СЛАУ совместна тогда и только тогда, когда \(\boldsymbol{\rank A=} \boldsymbol{\rank{}}\augm{\boldsymbol{A}}\), где \(\boldsymbol{A}\) --- основная матрица системы, а \(\augm{\boldsymbol{A}}\) --- расширенная матрица, полученная из основной матрицы приписыванием столбца с неоднородностями}
	\[\left\{\begin{array}{@{}l@{}}
		a_{11}x_{1}+\dotsb+a_{1n}x_{n}=b_{1},\\[3pt]
		a_{21}x_{1}+\dotsb+a_{2n}x_{n}=b_{2},\\[3pt]
		\hcdotsfor{1}\\[3pt]
		a_{m1}x_{1}+\dotsb+a_{mn}x_{n}=b_{n}.
	\end{array}\right.\]

	\[A=\begin{pmatrix}a_{11}&\cdots&a_{1n}\\ a_{21}&\cdots&a_{2n}\\ \hcdotsfor{3} \\ a_{m1}&\cdots&a_{mn}\end{pmatrix},\,\augm{A}=\begin{pmatrix}[ccc|c]a_{11}&\cdots&a_{1n}&b_{1}\\ a_{21}&\cdots&a_{2n}&b_{2}\\ \hcdotsfor{4}\\ a_{m1}&\cdots&a_{mn}&b_{n}\end{pmatrix}\]
	\begin{proof}\(\left< \Rightarrow \right>\) Предположим противное: система совместна, но \(\rank A<\rank\augm{A}\).
	
	Пусть \(\rank{A}=r\). Без потери общности предположим, что \(a_{11}\neq 0\) и. Приведём матрицу \(\augm{A}\) к верхнетреугольному (ступенчатому) виду.

	Получим \(\augm{A}\sim\begin{pmatrix}[ccccc|l]
										 a_{11}&\hcdotsfor{3}&a_{1n}	   &b_{1}		 \\
										 0	   &a'_{22}		 &\hcdotsfor{2}&a_{2n}&b'_{2}\\
									 	 \hcdotsfor{6}\\
								 	 	 0	   &\cdots       &a'_{rr}      &\cdots&a'_{rn}&b'_{r}\\
							 	 	 	 0	   &\cdots		 &0			   &\cdots&0	  &b'_{r+1}\end{pmatrix}\), где \(b'_{r+1}\neq 0\), т.к. \(\rank\augm{A}>\rank A\).\end{proof}
	%TODO: fix/check periods at the end of math structures
	%TODO: fix proof* and proof** (see \vect??)
	%TODO: fix/check newpages, "--" or "---"
	%TODO: fix spacing, and smth??
	%TODO: FIX FUCKING EVERYTHING THAT HAS BROKEN AS A RESULT OF REDEFINING MARGINS AND TIKZMARKEMPTY, ETC
\end{document}